
% Default to the notebook output style

    


% Inherit from the specified cell style.




    
\documentclass[11pt]{article}

    
    
    \usepackage[T1]{fontenc}
    % Nicer default font (+ math font) than Computer Modern for most use cases
    \usepackage{mathpazo}

    % Basic figure setup, for now with no caption control since it's done
    % automatically by Pandoc (which extracts ![](path) syntax from Markdown).
    \usepackage{graphicx}
    % We will generate all images so they have a width \maxwidth. This means
    % that they will get their normal width if they fit onto the page, but
    % are scaled down if they would overflow the margins.
    \makeatletter
    \def\maxwidth{\ifdim\Gin@nat@width>\linewidth\linewidth
    \else\Gin@nat@width\fi}
    \makeatother
    \let\Oldincludegraphics\includegraphics
    % Set max figure width to be 80% of text width, for now hardcoded.
    \renewcommand{\includegraphics}[1]{\Oldincludegraphics[width=.8\maxwidth]{#1}}
    % Ensure that by default, figures have no caption (until we provide a
    % proper Figure object with a Caption API and a way to capture that
    % in the conversion process - todo).
    \usepackage{caption}
    \DeclareCaptionLabelFormat{nolabel}{}
    \captionsetup{labelformat=nolabel}

    \usepackage{adjustbox} % Used to constrain images to a maximum size 
    \usepackage{xcolor} % Allow colors to be defined
    \usepackage{enumerate} % Needed for markdown enumerations to work
    \usepackage{geometry} % Used to adjust the document margins
    \usepackage{amsmath} % Equations
    \usepackage{amssymb} % Equations
    \usepackage{textcomp} % defines textquotesingle
    % Hack from http://tex.stackexchange.com/a/47451/13684:
    \AtBeginDocument{%
        \def\PYZsq{\textquotesingle}% Upright quotes in Pygmentized code
    }
    \usepackage{upquote} % Upright quotes for verbatim code
    \usepackage{eurosym} % defines \euro
    \usepackage[mathletters]{ucs} % Extended unicode (utf-8) support
    \usepackage[utf8x]{inputenc} % Allow utf-8 characters in the tex document
    \usepackage{fancyvrb} % verbatim replacement that allows latex
    \usepackage{grffile} % extends the file name processing of package graphics 
                         % to support a larger range 
    % The hyperref package gives us a pdf with properly built
    % internal navigation ('pdf bookmarks' for the table of contents,
    % internal cross-reference links, web links for URLs, etc.)
    \usepackage{hyperref}
    \usepackage{longtable} % longtable support required by pandoc >1.10
    \usepackage{booktabs}  % table support for pandoc > 1.12.2
    \usepackage[inline]{enumitem} % IRkernel/repr support (it uses the enumerate* environment)
    \usepackage[normalem]{ulem} % ulem is needed to support strikethroughs (\sout)
                                % normalem makes italics be italics, not underlines
    \usepackage{mathrsfs}
    

    
    
    % Colors for the hyperref package
    \definecolor{urlcolor}{rgb}{0,.145,.698}
    \definecolor{linkcolor}{rgb}{.71,0.21,0.01}
    \definecolor{citecolor}{rgb}{.12,.54,.11}

    % ANSI colors
    \definecolor{ansi-black}{HTML}{3E424D}
    \definecolor{ansi-black-intense}{HTML}{282C36}
    \definecolor{ansi-red}{HTML}{E75C58}
    \definecolor{ansi-red-intense}{HTML}{B22B31}
    \definecolor{ansi-green}{HTML}{00A250}
    \definecolor{ansi-green-intense}{HTML}{007427}
    \definecolor{ansi-yellow}{HTML}{DDB62B}
    \definecolor{ansi-yellow-intense}{HTML}{B27D12}
    \definecolor{ansi-blue}{HTML}{208FFB}
    \definecolor{ansi-blue-intense}{HTML}{0065CA}
    \definecolor{ansi-magenta}{HTML}{D160C4}
    \definecolor{ansi-magenta-intense}{HTML}{A03196}
    \definecolor{ansi-cyan}{HTML}{60C6C8}
    \definecolor{ansi-cyan-intense}{HTML}{258F8F}
    \definecolor{ansi-white}{HTML}{C5C1B4}
    \definecolor{ansi-white-intense}{HTML}{A1A6B2}
    \definecolor{ansi-default-inverse-fg}{HTML}{FFFFFF}
    \definecolor{ansi-default-inverse-bg}{HTML}{000000}

    % commands and environments needed by pandoc snippets
    % extracted from the output of `pandoc -s`
    \providecommand{\tightlist}{%
      \setlength{\itemsep}{0pt}\setlength{\parskip}{0pt}}
    \DefineVerbatimEnvironment{Highlighting}{Verbatim}{commandchars=\\\{\}}
    % Add ',fontsize=\small' for more characters per line
    \newenvironment{Shaded}{}{}
    \newcommand{\KeywordTok}[1]{\textcolor[rgb]{0.00,0.44,0.13}{\textbf{{#1}}}}
    \newcommand{\DataTypeTok}[1]{\textcolor[rgb]{0.56,0.13,0.00}{{#1}}}
    \newcommand{\DecValTok}[1]{\textcolor[rgb]{0.25,0.63,0.44}{{#1}}}
    \newcommand{\BaseNTok}[1]{\textcolor[rgb]{0.25,0.63,0.44}{{#1}}}
    \newcommand{\FloatTok}[1]{\textcolor[rgb]{0.25,0.63,0.44}{{#1}}}
    \newcommand{\CharTok}[1]{\textcolor[rgb]{0.25,0.44,0.63}{{#1}}}
    \newcommand{\StringTok}[1]{\textcolor[rgb]{0.25,0.44,0.63}{{#1}}}
    \newcommand{\CommentTok}[1]{\textcolor[rgb]{0.38,0.63,0.69}{\textit{{#1}}}}
    \newcommand{\OtherTok}[1]{\textcolor[rgb]{0.00,0.44,0.13}{{#1}}}
    \newcommand{\AlertTok}[1]{\textcolor[rgb]{1.00,0.00,0.00}{\textbf{{#1}}}}
    \newcommand{\FunctionTok}[1]{\textcolor[rgb]{0.02,0.16,0.49}{{#1}}}
    \newcommand{\RegionMarkerTok}[1]{{#1}}
    \newcommand{\ErrorTok}[1]{\textcolor[rgb]{1.00,0.00,0.00}{\textbf{{#1}}}}
    \newcommand{\NormalTok}[1]{{#1}}
    
    % Additional commands for more recent versions of Pandoc
    \newcommand{\ConstantTok}[1]{\textcolor[rgb]{0.53,0.00,0.00}{{#1}}}
    \newcommand{\SpecialCharTok}[1]{\textcolor[rgb]{0.25,0.44,0.63}{{#1}}}
    \newcommand{\VerbatimStringTok}[1]{\textcolor[rgb]{0.25,0.44,0.63}{{#1}}}
    \newcommand{\SpecialStringTok}[1]{\textcolor[rgb]{0.73,0.40,0.53}{{#1}}}
    \newcommand{\ImportTok}[1]{{#1}}
    \newcommand{\DocumentationTok}[1]{\textcolor[rgb]{0.73,0.13,0.13}{\textit{{#1}}}}
    \newcommand{\AnnotationTok}[1]{\textcolor[rgb]{0.38,0.63,0.69}{\textbf{\textit{{#1}}}}}
    \newcommand{\CommentVarTok}[1]{\textcolor[rgb]{0.38,0.63,0.69}{\textbf{\textit{{#1}}}}}
    \newcommand{\VariableTok}[1]{\textcolor[rgb]{0.10,0.09,0.49}{{#1}}}
    \newcommand{\ControlFlowTok}[1]{\textcolor[rgb]{0.00,0.44,0.13}{\textbf{{#1}}}}
    \newcommand{\OperatorTok}[1]{\textcolor[rgb]{0.40,0.40,0.40}{{#1}}}
    \newcommand{\BuiltInTok}[1]{{#1}}
    \newcommand{\ExtensionTok}[1]{{#1}}
    \newcommand{\PreprocessorTok}[1]{\textcolor[rgb]{0.74,0.48,0.00}{{#1}}}
    \newcommand{\AttributeTok}[1]{\textcolor[rgb]{0.49,0.56,0.16}{{#1}}}
    \newcommand{\InformationTok}[1]{\textcolor[rgb]{0.38,0.63,0.69}{\textbf{\textit{{#1}}}}}
    \newcommand{\WarningTok}[1]{\textcolor[rgb]{0.38,0.63,0.69}{\textbf{\textit{{#1}}}}}
    
    
    % Define a nice break command that doesn't care if a line doesn't already
    % exist.
    \def\br{\hspace*{\fill} \\* }
    % Math Jax compatibility definitions
    \def\gt{>}
    \def\lt{<}
    \let\Oldtex\TeX
    \let\Oldlatex\LaTeX
    \renewcommand{\TeX}{\textrm{\Oldtex}}
    \renewcommand{\LaTeX}{\textrm{\Oldlatex}}
    % Document parameters
    % Document title
    \title{DB0201EN-Week4-2-2-PeerAssign-v5-py}
    
    
    
    
    

    % Pygments definitions
    
\makeatletter
\def\PY@reset{\let\PY@it=\relax \let\PY@bf=\relax%
    \let\PY@ul=\relax \let\PY@tc=\relax%
    \let\PY@bc=\relax \let\PY@ff=\relax}
\def\PY@tok#1{\csname PY@tok@#1\endcsname}
\def\PY@toks#1+{\ifx\relax#1\empty\else%
    \PY@tok{#1}\expandafter\PY@toks\fi}
\def\PY@do#1{\PY@bc{\PY@tc{\PY@ul{%
    \PY@it{\PY@bf{\PY@ff{#1}}}}}}}
\def\PY#1#2{\PY@reset\PY@toks#1+\relax+\PY@do{#2}}

\expandafter\def\csname PY@tok@w\endcsname{\def\PY@tc##1{\textcolor[rgb]{0.73,0.73,0.73}{##1}}}
\expandafter\def\csname PY@tok@c\endcsname{\let\PY@it=\textit\def\PY@tc##1{\textcolor[rgb]{0.25,0.50,0.50}{##1}}}
\expandafter\def\csname PY@tok@cp\endcsname{\def\PY@tc##1{\textcolor[rgb]{0.74,0.48,0.00}{##1}}}
\expandafter\def\csname PY@tok@k\endcsname{\let\PY@bf=\textbf\def\PY@tc##1{\textcolor[rgb]{0.00,0.50,0.00}{##1}}}
\expandafter\def\csname PY@tok@kp\endcsname{\def\PY@tc##1{\textcolor[rgb]{0.00,0.50,0.00}{##1}}}
\expandafter\def\csname PY@tok@kt\endcsname{\def\PY@tc##1{\textcolor[rgb]{0.69,0.00,0.25}{##1}}}
\expandafter\def\csname PY@tok@o\endcsname{\def\PY@tc##1{\textcolor[rgb]{0.40,0.40,0.40}{##1}}}
\expandafter\def\csname PY@tok@ow\endcsname{\let\PY@bf=\textbf\def\PY@tc##1{\textcolor[rgb]{0.67,0.13,1.00}{##1}}}
\expandafter\def\csname PY@tok@nb\endcsname{\def\PY@tc##1{\textcolor[rgb]{0.00,0.50,0.00}{##1}}}
\expandafter\def\csname PY@tok@nf\endcsname{\def\PY@tc##1{\textcolor[rgb]{0.00,0.00,1.00}{##1}}}
\expandafter\def\csname PY@tok@nc\endcsname{\let\PY@bf=\textbf\def\PY@tc##1{\textcolor[rgb]{0.00,0.00,1.00}{##1}}}
\expandafter\def\csname PY@tok@nn\endcsname{\let\PY@bf=\textbf\def\PY@tc##1{\textcolor[rgb]{0.00,0.00,1.00}{##1}}}
\expandafter\def\csname PY@tok@ne\endcsname{\let\PY@bf=\textbf\def\PY@tc##1{\textcolor[rgb]{0.82,0.25,0.23}{##1}}}
\expandafter\def\csname PY@tok@nv\endcsname{\def\PY@tc##1{\textcolor[rgb]{0.10,0.09,0.49}{##1}}}
\expandafter\def\csname PY@tok@no\endcsname{\def\PY@tc##1{\textcolor[rgb]{0.53,0.00,0.00}{##1}}}
\expandafter\def\csname PY@tok@nl\endcsname{\def\PY@tc##1{\textcolor[rgb]{0.63,0.63,0.00}{##1}}}
\expandafter\def\csname PY@tok@ni\endcsname{\let\PY@bf=\textbf\def\PY@tc##1{\textcolor[rgb]{0.60,0.60,0.60}{##1}}}
\expandafter\def\csname PY@tok@na\endcsname{\def\PY@tc##1{\textcolor[rgb]{0.49,0.56,0.16}{##1}}}
\expandafter\def\csname PY@tok@nt\endcsname{\let\PY@bf=\textbf\def\PY@tc##1{\textcolor[rgb]{0.00,0.50,0.00}{##1}}}
\expandafter\def\csname PY@tok@nd\endcsname{\def\PY@tc##1{\textcolor[rgb]{0.67,0.13,1.00}{##1}}}
\expandafter\def\csname PY@tok@s\endcsname{\def\PY@tc##1{\textcolor[rgb]{0.73,0.13,0.13}{##1}}}
\expandafter\def\csname PY@tok@sd\endcsname{\let\PY@it=\textit\def\PY@tc##1{\textcolor[rgb]{0.73,0.13,0.13}{##1}}}
\expandafter\def\csname PY@tok@si\endcsname{\let\PY@bf=\textbf\def\PY@tc##1{\textcolor[rgb]{0.73,0.40,0.53}{##1}}}
\expandafter\def\csname PY@tok@se\endcsname{\let\PY@bf=\textbf\def\PY@tc##1{\textcolor[rgb]{0.73,0.40,0.13}{##1}}}
\expandafter\def\csname PY@tok@sr\endcsname{\def\PY@tc##1{\textcolor[rgb]{0.73,0.40,0.53}{##1}}}
\expandafter\def\csname PY@tok@ss\endcsname{\def\PY@tc##1{\textcolor[rgb]{0.10,0.09,0.49}{##1}}}
\expandafter\def\csname PY@tok@sx\endcsname{\def\PY@tc##1{\textcolor[rgb]{0.00,0.50,0.00}{##1}}}
\expandafter\def\csname PY@tok@m\endcsname{\def\PY@tc##1{\textcolor[rgb]{0.40,0.40,0.40}{##1}}}
\expandafter\def\csname PY@tok@gh\endcsname{\let\PY@bf=\textbf\def\PY@tc##1{\textcolor[rgb]{0.00,0.00,0.50}{##1}}}
\expandafter\def\csname PY@tok@gu\endcsname{\let\PY@bf=\textbf\def\PY@tc##1{\textcolor[rgb]{0.50,0.00,0.50}{##1}}}
\expandafter\def\csname PY@tok@gd\endcsname{\def\PY@tc##1{\textcolor[rgb]{0.63,0.00,0.00}{##1}}}
\expandafter\def\csname PY@tok@gi\endcsname{\def\PY@tc##1{\textcolor[rgb]{0.00,0.63,0.00}{##1}}}
\expandafter\def\csname PY@tok@gr\endcsname{\def\PY@tc##1{\textcolor[rgb]{1.00,0.00,0.00}{##1}}}
\expandafter\def\csname PY@tok@ge\endcsname{\let\PY@it=\textit}
\expandafter\def\csname PY@tok@gs\endcsname{\let\PY@bf=\textbf}
\expandafter\def\csname PY@tok@gp\endcsname{\let\PY@bf=\textbf\def\PY@tc##1{\textcolor[rgb]{0.00,0.00,0.50}{##1}}}
\expandafter\def\csname PY@tok@go\endcsname{\def\PY@tc##1{\textcolor[rgb]{0.53,0.53,0.53}{##1}}}
\expandafter\def\csname PY@tok@gt\endcsname{\def\PY@tc##1{\textcolor[rgb]{0.00,0.27,0.87}{##1}}}
\expandafter\def\csname PY@tok@err\endcsname{\def\PY@bc##1{\setlength{\fboxsep}{0pt}\fcolorbox[rgb]{1.00,0.00,0.00}{1,1,1}{\strut ##1}}}
\expandafter\def\csname PY@tok@kc\endcsname{\let\PY@bf=\textbf\def\PY@tc##1{\textcolor[rgb]{0.00,0.50,0.00}{##1}}}
\expandafter\def\csname PY@tok@kd\endcsname{\let\PY@bf=\textbf\def\PY@tc##1{\textcolor[rgb]{0.00,0.50,0.00}{##1}}}
\expandafter\def\csname PY@tok@kn\endcsname{\let\PY@bf=\textbf\def\PY@tc##1{\textcolor[rgb]{0.00,0.50,0.00}{##1}}}
\expandafter\def\csname PY@tok@kr\endcsname{\let\PY@bf=\textbf\def\PY@tc##1{\textcolor[rgb]{0.00,0.50,0.00}{##1}}}
\expandafter\def\csname PY@tok@bp\endcsname{\def\PY@tc##1{\textcolor[rgb]{0.00,0.50,0.00}{##1}}}
\expandafter\def\csname PY@tok@fm\endcsname{\def\PY@tc##1{\textcolor[rgb]{0.00,0.00,1.00}{##1}}}
\expandafter\def\csname PY@tok@vc\endcsname{\def\PY@tc##1{\textcolor[rgb]{0.10,0.09,0.49}{##1}}}
\expandafter\def\csname PY@tok@vg\endcsname{\def\PY@tc##1{\textcolor[rgb]{0.10,0.09,0.49}{##1}}}
\expandafter\def\csname PY@tok@vi\endcsname{\def\PY@tc##1{\textcolor[rgb]{0.10,0.09,0.49}{##1}}}
\expandafter\def\csname PY@tok@vm\endcsname{\def\PY@tc##1{\textcolor[rgb]{0.10,0.09,0.49}{##1}}}
\expandafter\def\csname PY@tok@sa\endcsname{\def\PY@tc##1{\textcolor[rgb]{0.73,0.13,0.13}{##1}}}
\expandafter\def\csname PY@tok@sb\endcsname{\def\PY@tc##1{\textcolor[rgb]{0.73,0.13,0.13}{##1}}}
\expandafter\def\csname PY@tok@sc\endcsname{\def\PY@tc##1{\textcolor[rgb]{0.73,0.13,0.13}{##1}}}
\expandafter\def\csname PY@tok@dl\endcsname{\def\PY@tc##1{\textcolor[rgb]{0.73,0.13,0.13}{##1}}}
\expandafter\def\csname PY@tok@s2\endcsname{\def\PY@tc##1{\textcolor[rgb]{0.73,0.13,0.13}{##1}}}
\expandafter\def\csname PY@tok@sh\endcsname{\def\PY@tc##1{\textcolor[rgb]{0.73,0.13,0.13}{##1}}}
\expandafter\def\csname PY@tok@s1\endcsname{\def\PY@tc##1{\textcolor[rgb]{0.73,0.13,0.13}{##1}}}
\expandafter\def\csname PY@tok@mb\endcsname{\def\PY@tc##1{\textcolor[rgb]{0.40,0.40,0.40}{##1}}}
\expandafter\def\csname PY@tok@mf\endcsname{\def\PY@tc##1{\textcolor[rgb]{0.40,0.40,0.40}{##1}}}
\expandafter\def\csname PY@tok@mh\endcsname{\def\PY@tc##1{\textcolor[rgb]{0.40,0.40,0.40}{##1}}}
\expandafter\def\csname PY@tok@mi\endcsname{\def\PY@tc##1{\textcolor[rgb]{0.40,0.40,0.40}{##1}}}
\expandafter\def\csname PY@tok@il\endcsname{\def\PY@tc##1{\textcolor[rgb]{0.40,0.40,0.40}{##1}}}
\expandafter\def\csname PY@tok@mo\endcsname{\def\PY@tc##1{\textcolor[rgb]{0.40,0.40,0.40}{##1}}}
\expandafter\def\csname PY@tok@ch\endcsname{\let\PY@it=\textit\def\PY@tc##1{\textcolor[rgb]{0.25,0.50,0.50}{##1}}}
\expandafter\def\csname PY@tok@cm\endcsname{\let\PY@it=\textit\def\PY@tc##1{\textcolor[rgb]{0.25,0.50,0.50}{##1}}}
\expandafter\def\csname PY@tok@cpf\endcsname{\let\PY@it=\textit\def\PY@tc##1{\textcolor[rgb]{0.25,0.50,0.50}{##1}}}
\expandafter\def\csname PY@tok@c1\endcsname{\let\PY@it=\textit\def\PY@tc##1{\textcolor[rgb]{0.25,0.50,0.50}{##1}}}
\expandafter\def\csname PY@tok@cs\endcsname{\let\PY@it=\textit\def\PY@tc##1{\textcolor[rgb]{0.25,0.50,0.50}{##1}}}

\def\PYZbs{\char`\\}
\def\PYZus{\char`\_}
\def\PYZob{\char`\{}
\def\PYZcb{\char`\}}
\def\PYZca{\char`\^}
\def\PYZam{\char`\&}
\def\PYZlt{\char`\<}
\def\PYZgt{\char`\>}
\def\PYZsh{\char`\#}
\def\PYZpc{\char`\%}
\def\PYZdl{\char`\$}
\def\PYZhy{\char`\-}
\def\PYZsq{\char`\'}
\def\PYZdq{\char`\"}
\def\PYZti{\char`\~}
% for compatibility with earlier versions
\def\PYZat{@}
\def\PYZlb{[}
\def\PYZrb{]}
\makeatother


    % Exact colors from NB
    \definecolor{incolor}{rgb}{0.0, 0.0, 0.5}
    \definecolor{outcolor}{rgb}{0.545, 0.0, 0.0}



    
    % Prevent overflowing lines due to hard-to-break entities
    \sloppy 
    % Setup hyperref package
    \hypersetup{
      breaklinks=true,  % so long urls are correctly broken across lines
      colorlinks=true,
      urlcolor=urlcolor,
      linkcolor=linkcolor,
      citecolor=citecolor,
      }
    % Slightly bigger margins than the latex defaults
    
    \geometry{verbose,tmargin=1in,bmargin=1in,lmargin=1in,rmargin=1in}
    
    

    \begin{document}
    
    
    \maketitle
    
    

    
    Assignment: Notebook for Peer Assignment

    \section{Introduction}\label{introduction}

Using this Python notebook you will: 1. Understand 3 Chicago datasets\\
1. Load the 3 datasets into 3 tables in a Db2 database 1. Execute SQL
queries to answer assignment questions

    \subsection{Understand the datasets}\label{understand-the-datasets}

To complete the assignment problems in this notebook you will be using
three datasets that are available on the city of Chicago's Data Portal:
1. Socioeconomic Indicators in Chicago 1. Chicago Public Schools 1.
Chicago Crime Data

\subsubsection{1. Socioeconomic Indicators in
Chicago}\label{socioeconomic-indicators-in-chicago}

This dataset contains a selection of six socioeconomic indicators of
public health significance and a ``hardship index,'' for each Chicago
community area, for the years 2008 -- 2012.

For this assignment you will use a snapshot of this dataset which can be
downloaded from:
https://ibm.box.com/shared/static/05c3415cbfbtfnr2fx4atenb2sd361ze.csv

A detailed description of this dataset and the original dataset can be
obtained from the Chicago Data Portal at:
https://data.cityofchicago.org/Health-Human-Services/Census-Data-Selected-socioeconomic-indicators-in-C/kn9c-c2s2

\subsubsection{2. Chicago Public Schools}\label{chicago-public-schools}

This dataset shows all school level performance data used to create CPS
School Report Cards for the 2011-2012 school year. This dataset is
provided by the city of Chicago's Data Portal.

For this assignment you will use a snapshot of this dataset which can be
downloaded from:
https://ibm.box.com/shared/static/f9gjvj1gjmxxzycdhplzt01qtz0s7ew7.csv

A detailed description of this dataset and the original dataset can be
obtained from the Chicago Data Portal at:
https://data.cityofchicago.org/Education/Chicago-Public-Schools-Progress-Report-Cards-2011-/9xs2-f89t

\subsubsection{3. Chicago Crime Data}\label{chicago-crime-data}

This dataset reflects reported incidents of crime (with the exception of
murders where data exists for each victim) that occurred in the City of
Chicago from 2001 to present, minus the most recent seven days.

This dataset is quite large - over 1.5GB in size with over 6.5 million
rows. For the purposes of this assignment we will use a much smaller
sample of this dataset which can be downloaded from:
https://ibm.box.com/shared/static/svflyugsr9zbqy5bmowgswqemfpm1x7f.csv

A detailed description of this dataset and the original dataset can be
obtained from the Chicago Data Portal at:
https://data.cityofchicago.org/Public-Safety/Crimes-2001-to-present/ijzp-q8t2

    \subsubsection{Download the datasets}\label{download-the-datasets}

In many cases the dataset to be analyzed is available as a .CSV (comma
separated values) file, perhaps on the internet. Click on the links
below to download and save the datasets (.CSV files): 1.
\textbf{CENSUS\_DATA:}
https://ibm.box.com/shared/static/05c3415cbfbtfnr2fx4atenb2sd361ze.csv
1. \textbf{CHICAGO\_PUBLIC\_SCHOOLS}
https://ibm.box.com/shared/static/f9gjvj1gjmxxzycdhplzt01qtz0s7ew7.csv
1. \textbf{CHICAGO\_CRIME\_DATA:}
https://ibm.box.com/shared/static/svflyugsr9zbqy5bmowgswqemfpm1x7f.csv

\textbf{NOTE:} Ensure you have downloaded the datasets using the links
above instead of directly from the Chicago Data Portal. The versions
linked here are subsets of the original datasets and have some of the
column names modified to be more database friendly which will make it
easier to complete this assignment.

    \subsubsection{Store the datasets in database
tables}\label{store-the-datasets-in-database-tables}

To analyze the data using SQL, it first needs to be stored in the
database.

While it is easier to read the dataset into a Pandas dataframe and then
PERSIST it into the database as we saw in Week 3 Lab 3, it results in
mapping to default datatypes which may not be optimal for SQL querying.
For example a long textual field may map to a CLOB instead of a VARCHAR.

Therefore, \textbf{it is highly recommended to manually load the table
using the database console LOAD tool, as indicated in Week 2 Lab 1 Part
II}. The only difference with that lab is that in Step 5 of the
instructions you will need to click on create "(+) New Table" and
specify the name of the table you want to create and then click "Next".

\subparagraph{Now open the Db2 console, open the LOAD tool, Select /
Drag the .CSV file for the first dataset, Next create a New Table, and
then follow the steps on-screen instructions to load the data. Name the
new tables as
folows:}\label{now-open-the-db2-console-open-the-load-tool-select-drag-the-.csv-file-for-the-first-dataset-next-create-a-new-table-and-then-follow-the-steps-on-screen-instructions-to-load-the-data.-name-the-new-tables-as-folows}

\begin{enumerate}
\def\labelenumi{\arabic{enumi}.}
\tightlist
\item
  \textbf{CENSUS\_DATA}
\item
  \textbf{CHICAGO\_PUBLIC\_SCHOOLS}
\item
  \textbf{CHICAGO\_CRIME\_DATA}
\end{enumerate}

    \subsubsection{Connect to the database}\label{connect-to-the-database}

Let us first load the SQL extension and establish a connection with the
database

    \begin{Verbatim}[commandchars=\\\{\}]
{\color{incolor}In [{\color{incolor}1}]:} \PY{o}{\PYZpc{}}\PY{k}{load\PYZus{}ext} sql
\end{Verbatim}

    In the next cell enter your db2 connection string. Recall you created
Service Credentials for your Db2 instance in first lab in Week 3. From
the \textbf{uri} field of your Db2 service credentials copy everything
after db2:// (except the double quote at the end) and paste it in the
cell below after ibm\_db\_sa://

    \begin{Verbatim}[commandchars=\\\{\}]
{\color{incolor}In [{\color{incolor}2}]:} \PY{c+c1}{\PYZsh{} Remember the connection string is of the format:}
        \PY{c+c1}{\PYZsh{} \PYZpc{}sql ibm\PYZus{}db\PYZus{}sa://my\PYZhy{}username:my\PYZhy{}password@my\PYZhy{}hostname:my\PYZhy{}port/my\PYZhy{}db\PYZhy{}name}
        \PY{c+c1}{\PYZsh{} Enter the connection string for your Db2 on Cloud database instance below}
        \PY{o}{\PYZpc{}}\PY{k}{sql} ibm\PYZus{}db\PYZus{}sa://lln32654:69d16lp9\PYZpc{}5Ecw5lj4g@dashdb\PYZhy{}txn\PYZhy{}sbox\PYZhy{}yp\PYZhy{}dal09\PYZhy{}03.services.dal.bluemix.net:50000/BLUDB
\end{Verbatim}

\begin{Verbatim}[commandchars=\\\{\}]
{\color{outcolor}Out[{\color{outcolor}2}]:} 'Connected: lln32654@BLUDB'
\end{Verbatim}
            
    \subsection{Problems}\label{problems}

Now write and execute SQL queries to solve assignment problems

\subsubsection{Problem 1}\label{problem-1}

\subparagraph{Find the total number of crimes recorded in the CRIME
table}\label{find-the-total-number-of-crimes-recorded-in-the-crime-table}

    \begin{Verbatim}[commandchars=\\\{\}]
{\color{incolor}In [{\color{incolor}40}]:} \PY{c+c1}{\PYZsh{} Rows in Crime table}
         \PY{o}{\PYZpc{}}\PY{k}{sql} select count(*) AS \PYZdq{}Number\PYZus{}of\PYZus{}Crimes\PYZdq{} from chicago\PYZus{}crime\PYZus{}data
\end{Verbatim}

    \begin{Verbatim}[commandchars=\\\{\}]
 * ibm\_db\_sa://lln32654:***@dashdb-txn-sbox-yp-dal09-03.services.dal.bluemix.net:50000/BLUDB
Done.

    \end{Verbatim}

\begin{Verbatim}[commandchars=\\\{\}]
{\color{outcolor}Out[{\color{outcolor}40}]:} [(Decimal('533'),)]
\end{Verbatim}
            
    \subsubsection{Problem 2}\label{problem-2}

\subparagraph{Retrieve first 10 rows from the CRIME
table}\label{retrieve-first-10-rows-from-the-crime-table}

    \begin{Verbatim}[commandchars=\\\{\}]
{\color{incolor}In [{\color{incolor}4}]:} \PY{o}{\PYZpc{}}\PY{k}{sql} select * from chicago\PYZus{}crime\PYZus{}data \PYZbs{}
        \PY{n}{fetch} \PY{n}{first} \PY{l+m+mi}{10} \PY{n}{rows} \PY{n}{only}
\end{Verbatim}

    \begin{Verbatim}[commandchars=\\\{\}]
 * ibm\_db\_sa://lln32654:***@dashdb-txn-sbox-yp-dal09-03.services.dal.bluemix.net:50000/BLUDB
Done.

    \end{Verbatim}

\begin{Verbatim}[commandchars=\\\{\}]
{\color{outcolor}Out[{\color{outcolor}4}]:} [(3512276, 'HK587712', '08/28/2004 05:50:56 PM', '047XX S KEDZIE AVE', '890', 'THEFT', 'FROM BUILDING', 'SMALL RETAIL STORE', 'FALSE', 'FALSE', 911, 9, 14, 58, '6', 1155838, 1873050, 2004, '02/10/2018 03:50:01 PM', Decimal('41.80744050'), Decimal('-87.70395585'), '(41.8074405, -87.703955849)'),
         (3406613, 'HK456306', '06/26/2004 12:40:00 PM', '009XX N CENTRAL PARK AVE', '820', 'THEFT', '\$500 AND UNDER', 'OTHER', 'FALSE', 'FALSE', 1112, 11, 27, 23, '6', 1152206, 1906127, 2004, '02/28/2018 03:56:25 PM', Decimal('41.89827996'), Decimal('-87.71640551'), '(41.898279962, -87.716405505)'),
         (8002131, 'HT233595', '04/04/2011 05:45:00 AM', '043XX S WABASH AVE', '820', 'THEFT', '\$500 AND UNDER', 'NURSING HOME/RETIREMENT HOME', 'FALSE', 'FALSE', 221, 2, 3, 38, '6', 1177436, 1876313, 2011, '02/10/2018 03:50:01 PM', Decimal('41.81593313'), Decimal('-87.62464213'), '(41.815933131, -87.624642127)'),
         (7903289, 'HT133522', '12/30/2010 04:30:00 PM', '083XX S KINGSTON AVE', '840', 'THEFT', 'FINANCIAL ID THEFT: OVER \$300', 'RESIDENCE', 'FALSE', 'FALSE', 423, 4, 7, 46, '6', 1194622, 1850125, 2010, '02/10/2018 03:50:01 PM', Decimal('41.74366532'), Decimal('-87.56246276'), '(41.743665322, -87.562462756)'),
         (10402076, 'HZ138551', '02/02/2016 07:30:00 PM', '033XX W 66TH ST', '820', 'THEFT', '\$500 AND UNDER', 'ALLEY', 'FALSE', 'FALSE', 831, 8, 15, 66, '6', 1155240, 1860661, 2016, '02/10/2018 03:50:01 PM', Decimal('41.77345530'), Decimal('-87.70648047'), '(41.773455295, -87.706480471)'),
         (7732712, 'HS540106', '09/29/2010 07:59:00 AM', '006XX W CHICAGO AVE', '810', 'THEFT', 'OVER \$500', 'PARKING LOT/GARAGE(NON.RESID.)', 'FALSE', 'FALSE', 1323, 12, 27, 24, '6', 1171668, 1905607, 2010, '02/10/2018 03:50:01 PM', Decimal('41.89644677'), Decimal('-87.64493868'), '(41.896446772, -87.644938678)'),
         (10769475, 'HZ534771', '11/30/2016 01:15:00 AM', '050XX N KEDZIE AVE', '810', 'THEFT', 'OVER \$500', 'STREET', 'FALSE', 'FALSE', 1713, 17, 33, 14, '6', 1154133, 1933314, 2016, '02/10/2018 03:50:01 PM', Decimal('41.97284491'), Decimal('-87.70860008'), '(41.972844913, -87.708600079)'),
         (4494340, 'HL793243', '12/16/2005 04:45:00 PM', '005XX E PERSHING RD', '860', 'THEFT', 'RETAIL THEFT', 'GROCERY FOOD STORE', 'TRUE', 'FALSE', 213, 2, 3, 38, '6', 1180448, 1879234, 2005, '02/28/2018 03:56:25 PM', Decimal('41.82387989'), Decimal('-87.61350386'), '(41.823879885, -87.613503857)'),
         (3778925, 'HL149610', '01/28/2005 05:00:00 PM', '100XX S WASHTENAW AVE', '810', 'THEFT', 'OVER \$500', 'STREET', 'FALSE', 'FALSE', 2211, 22, 19, 72, '6', 1160129, 1838040, 2005, '02/28/2018 03:56:25 PM', Decimal('41.71128051'), Decimal('-87.68917910'), '(41.711280513, -87.689179097)'),
         (3324217, 'HK361551', '05/13/2004 02:15:00 PM', '033XX W BELMONT AVE', '820', 'THEFT', '\$500 AND UNDER', 'SMALL RETAIL STORE', 'FALSE', 'FALSE', 1733, 17, 35, 21, '6', 1153590, 1921084, 2004, '02/28/2018 03:56:25 PM', Decimal('41.93929582'), Decimal('-87.71092344'), '(41.939295821, -87.710923442)')]
\end{Verbatim}
            
    \subsubsection{Problem 3}\label{problem-3}

\subparagraph{How many crimes involve an
arrest?}\label{how-many-crimes-involve-an-arrest}

    \begin{Verbatim}[commandchars=\\\{\}]
{\color{incolor}In [{\color{incolor}41}]:} \PY{o}{\PYZpc{}}\PY{k}{sql} select count(*) AS \PYZdq{}Number\PYZus{}of\PYZus{}Crimes\PYZus{}Involving\PYZus{}an\PYZus{}Arrest\PYZdq{} from chicago\PYZus{}crime\PYZus{}data where arrest = \PYZsq{}TRUE\PYZsq{}
\end{Verbatim}

    \begin{Verbatim}[commandchars=\\\{\}]
 * ibm\_db\_sa://lln32654:***@dashdb-txn-sbox-yp-dal09-03.services.dal.bluemix.net:50000/BLUDB
Done.

    \end{Verbatim}

\begin{Verbatim}[commandchars=\\\{\}]
{\color{outcolor}Out[{\color{outcolor}41}]:} [(Decimal('163'),)]
\end{Verbatim}
            
    \subsubsection{Problem 4}\label{problem-4}

\subparagraph{Which unique types of crimes have been recorded at GAS
STATION
locations?}\label{which-unique-types-of-crimes-have-been-recorded-at-gas-station-locations}

    \begin{Verbatim}[commandchars=\\\{\}]
{\color{incolor}In [{\color{incolor}6}]:} \PY{o}{\PYZpc{}}\PY{k}{sql} select distinct primary\PYZus{}type, location\PYZus{}description from chicago\PYZus{}crime\PYZus{}data where location\PYZus{}description = \PYZsq{}GAS STATION\PYZsq{}
\end{Verbatim}

    \begin{Verbatim}[commandchars=\\\{\}]
 * ibm\_db\_sa://lln32654:***@dashdb-txn-sbox-yp-dal09-03.services.dal.bluemix.net:50000/BLUDB
Done.

    \end{Verbatim}

\begin{Verbatim}[commandchars=\\\{\}]
{\color{outcolor}Out[{\color{outcolor}6}]:} [('CRIMINAL TRESPASS', 'GAS STATION'),
         ('NARCOTICS', 'GAS STATION'),
         ('ROBBERY', 'GAS STATION'),
         ('THEFT', 'GAS STATION')]
\end{Verbatim}
            
    Hint: Which column lists types of crimes e.g. THEFT?

    \subsubsection{Problem 5}\label{problem-5}

\subparagraph{In the CENUS\_DATA table list all Community Areas whose
names start with the letter
`B'.}\label{in-the-cenus_data-table-list-all-community-areas-whose-names-start-with-the-letter-b.}

    \begin{Verbatim}[commandchars=\\\{\}]
{\color{incolor}In [{\color{incolor}7}]:} \PY{o}{\PYZpc{}}\PY{k}{sql} select community\PYZus{}area\PYZus{}name from census\PYZus{}data where census\PYZus{}data.community\PYZus{}area\PYZus{}name LIKE \PYZsq{}B\PYZpc{}\PYZsq{}
\end{Verbatim}

    \begin{Verbatim}[commandchars=\\\{\}]
 * ibm\_db\_sa://lln32654:***@dashdb-txn-sbox-yp-dal09-03.services.dal.bluemix.net:50000/BLUDB
Done.

    \end{Verbatim}

\begin{Verbatim}[commandchars=\\\{\}]
{\color{outcolor}Out[{\color{outcolor}7}]:} [('Belmont Cragin',),
         ('Burnside',),
         ('Brighton Park',),
         ('Bridgeport',),
         ('Beverly',)]
\end{Verbatim}
            
    \subsubsection{Problem 6}\label{problem-6}

\subparagraph{Which schools in Community Areas 10 to 15 are healthy
school
certified?}\label{which-schools-in-community-areas-10-to-15-are-healthy-school-certified}

    \begin{Verbatim}[commandchars=\\\{\}]
{\color{incolor}In [{\color{incolor}33}]:} \PY{o}{\PYZpc{}}\PY{k}{sql} select name\PYZus{}of\PYZus{}school, community\PYZus{}area\PYZus{}number, healthy\PYZus{}school\PYZus{}certified from CHICAGO\PYZus{}PUBLIC\PYZus{}SCHOOLS WHERE (community\PYZus{}area\PYZus{}number between 10 and 15) AND healthy\PYZus{}school\PYZus{}certified =\PYZsq{}Yes\PYZsq{}
\end{Verbatim}

    \begin{Verbatim}[commandchars=\\\{\}]
 * ibm\_db\_sa://lln32654:***@dashdb-txn-sbox-yp-dal09-03.services.dal.bluemix.net:50000/BLUDB
Done.

    \end{Verbatim}

\begin{Verbatim}[commandchars=\\\{\}]
{\color{outcolor}Out[{\color{outcolor}33}]:} [('Rufus M Hitch Elementary School', 10, 'Yes')]
\end{Verbatim}
            
    \subsubsection{Problem 7}\label{problem-7}

\subparagraph{What is the average school Safety
Score?}\label{what-is-the-average-school-safety-score}

    \begin{Verbatim}[commandchars=\\\{\}]
{\color{incolor}In [{\color{incolor}50}]:} \PY{o}{\PYZpc{}}\PY{k}{sql} select round(avg(safety\PYZus{}score), 2) AS \PYZdq{}Average\PYZus{}Safety\PYZus{}Score\PYZdq{} from CHICAGO\PYZus{}PUBLIC\PYZus{}SCHOOLS
\end{Verbatim}

    \begin{Verbatim}[commandchars=\\\{\}]
 * ibm\_db\_sa://lln32654:***@dashdb-txn-sbox-yp-dal09-03.services.dal.bluemix.net:50000/BLUDB
Done.

    \end{Verbatim}

\begin{Verbatim}[commandchars=\\\{\}]
{\color{outcolor}Out[{\color{outcolor}50}]:} [(Decimal('49.500000'),)]
\end{Verbatim}
            
    \subsubsection{Problem 8}\label{problem-8}

\subparagraph{List the top 5 Community Areas by average College
Enrollment {[}number of
students{]}}\label{list-the-top-5-community-areas-by-average-college-enrollment-number-of-students}

    \begin{Verbatim}[commandchars=\\\{\}]
{\color{incolor}In [{\color{incolor}87}]:} \PY{o}{\PYZpc{}}\PY{k}{sql} select COMMUNITY\PYZus{}AREA\PYZus{}NAME, round(avg(COLLEGE\PYZus{}ENROLLMENT),2) as Average\PYZus{}College\PYZus{}Enrollment from CHICAGO\PYZus{}PUBLIC\PYZus{}SCHOOLS\PYZbs{}
         \PY{n}{group} \PY{n}{by} \PY{n}{COMMUNITY\PYZus{}AREA\PYZus{}NAME} \PYZbs{}
         \PY{n}{ORDER} \PY{n}{BY} \PY{n}{Average\PYZus{}College\PYZus{}Enrollment} \PY{n}{DESC}\PYZbs{}
         \PY{n}{fetch} \PY{n}{first} \PY{l+m+mi}{5} \PY{n}{rows} \PY{n}{only}
\end{Verbatim}

    \begin{Verbatim}[commandchars=\\\{\}]
 * ibm\_db\_sa://lln32654:***@dashdb-txn-sbox-yp-dal09-03.services.dal.bluemix.net:50000/BLUDB
Done.

    \end{Verbatim}

\begin{Verbatim}[commandchars=\\\{\}]
{\color{outcolor}Out[{\color{outcolor}87}]:} [('ARCHER HEIGHTS', Decimal('2411.500000')),
          ('MONTCLARE', Decimal('1317.000000')),
          ('WEST ELSDON', Decimal('1233.330000')),
          ('BRIGHTON PARK', Decimal('1205.880000')),
          ('BELMONT CRAGIN', Decimal('1198.830000'))]
\end{Verbatim}
            
    \subsubsection{Problem 9}\label{problem-9}

\subparagraph{Use a sub-query to determine which Community Area has the
least value for school Safety
Score?}\label{use-a-sub-query-to-determine-which-community-area-has-the-least-value-for-school-safety-score}

    \begin{Verbatim}[commandchars=\\\{\}]
{\color{incolor}In [{\color{incolor}92}]:} \PY{o}{\PYZpc{}}\PY{k}{sql} select COMMUNITY\PYZus{}AREA\PYZus{}NAME, SAFETY\PYZus{}SCORE from CHICAGO\PYZus{}PUBLIC\PYZus{}SCHOOLS WHERE\PYZbs{}
         \PY{n}{SAFETY\PYZus{}SCORE} \PY{o}{=} \PY{p}{(}\PY{n}{SELECT} \PY{n}{MIN}\PY{p}{(}\PY{n}{SAFETY\PYZus{}SCORE}\PY{p}{)} \PY{n}{FROM} \PY{n}{CHICAGO\PYZus{}PUBLIC\PYZus{}SCHOOLS}\PY{p}{)}
\end{Verbatim}

    \begin{Verbatim}[commandchars=\\\{\}]
 * ibm\_db\_sa://lln32654:***@dashdb-txn-sbox-yp-dal09-03.services.dal.bluemix.net:50000/BLUDB
Done.

    \end{Verbatim}

\begin{Verbatim}[commandchars=\\\{\}]
{\color{outcolor}Out[{\color{outcolor}92}]:} [('WASHINGTON PARK', 1)]
\end{Verbatim}
            
    \subsubsection{Problem 10}\label{problem-10}

\subparagraph{{[}Without using an explicit JOIN operator{]} Find the Per
Capita Income of the Community Area which has a school Safety Score of
1.}\label{without-using-an-explicit-join-operator-find-the-per-capita-income-of-the-community-area-which-has-a-school-safety-score-of-1.}

    \begin{Verbatim}[commandchars=\\\{\}]
{\color{incolor}In [{\color{incolor}196}]:} \PY{c+c1}{\PYZsh{}}
          \PY{o}{\PYZpc{}}\PY{k}{sql} select community\PYZus{}area\PYZus{}number,PER\PYZus{}CAPITA\PYZus{}INCOME\PYZbs{}
          \PY{k+kn}{from} \PY{n+nn}{CENSUS\PYZus{}DATA} \PY{n}{where} \PY{n}{community\PYZus{}area\PYZus{}number} \PY{o+ow}{in}\PYZbs{}
          \PY{p}{(}\PY{n}{select} \PY{n}{community\PYZus{}area\PYZus{}number} \PY{k+kn}{from} \PY{n+nn}{CHICAGO\PYZus{}PUBLIC\PYZus{}SCHOOLS} \PY{n}{where} \PY{n}{SAFETY\PYZus{}SCORE}\PY{o}{=}\PY{l+m+mi}{1}\PY{p}{)}
\end{Verbatim}

    \begin{Verbatim}[commandchars=\\\{\}]
 * ibm\_db\_sa://lln32654:***@dashdb-txn-sbox-yp-dal09-03.services.dal.bluemix.net:50000/BLUDB
Done.

    \end{Verbatim}

\begin{Verbatim}[commandchars=\\\{\}]
{\color{outcolor}Out[{\color{outcolor}196}]:} [(40, 13785)]
\end{Verbatim}
            
    Copyright © 2018
\href{cognitiveclass.ai?utm_source=bducopyrightlink\&utm_medium=dswb\&utm_campaign=bdu}{cognitiveclass.ai}.
This notebook and its source code are released under the terms of the
\href{https://bigdatauniversity.com/mit-license/}{MIT License}.


    % Add a bibliography block to the postdoc
    
    
    
    \end{document}
