
% Default to the notebook output style

    


% Inherit from the specified cell style.




    
\documentclass[11pt]{article}

    
    
    \usepackage[T1]{fontenc}
    % Nicer default font (+ math font) than Computer Modern for most use cases
    \usepackage{mathpazo}

    % Basic figure setup, for now with no caption control since it's done
    % automatically by Pandoc (which extracts ![](path) syntax from Markdown).
    \usepackage{graphicx}
    % We will generate all images so they have a width \maxwidth. This means
    % that they will get their normal width if they fit onto the page, but
    % are scaled down if they would overflow the margins.
    \makeatletter
    \def\maxwidth{\ifdim\Gin@nat@width>\linewidth\linewidth
    \else\Gin@nat@width\fi}
    \makeatother
    \let\Oldincludegraphics\includegraphics
    % Set max figure width to be 80% of text width, for now hardcoded.
    \renewcommand{\includegraphics}[1]{\Oldincludegraphics[width=.8\maxwidth]{#1}}
    % Ensure that by default, figures have no caption (until we provide a
    % proper Figure object with a Caption API and a way to capture that
    % in the conversion process - todo).
    \usepackage{caption}
    \DeclareCaptionLabelFormat{nolabel}{}
    \captionsetup{labelformat=nolabel}

    \usepackage{adjustbox} % Used to constrain images to a maximum size 
    \usepackage{xcolor} % Allow colors to be defined
    \usepackage{enumerate} % Needed for markdown enumerations to work
    \usepackage{geometry} % Used to adjust the document margins
    \usepackage{amsmath} % Equations
    \usepackage{amssymb} % Equations
    \usepackage{textcomp} % defines textquotesingle
    % Hack from http://tex.stackexchange.com/a/47451/13684:
    \AtBeginDocument{%
        \def\PYZsq{\textquotesingle}% Upright quotes in Pygmentized code
    }
    \usepackage{upquote} % Upright quotes for verbatim code
    \usepackage{eurosym} % defines \euro
    \usepackage[mathletters]{ucs} % Extended unicode (utf-8) support
    \usepackage[utf8x]{inputenc} % Allow utf-8 characters in the tex document
    \usepackage{fancyvrb} % verbatim replacement that allows latex
    \usepackage{grffile} % extends the file name processing of package graphics 
                         % to support a larger range 
    % The hyperref package gives us a pdf with properly built
    % internal navigation ('pdf bookmarks' for the table of contents,
    % internal cross-reference links, web links for URLs, etc.)
    \usepackage{hyperref}
    \usepackage{longtable} % longtable support required by pandoc >1.10
    \usepackage{booktabs}  % table support for pandoc > 1.12.2
    \usepackage[inline]{enumitem} % IRkernel/repr support (it uses the enumerate* environment)
    \usepackage[normalem]{ulem} % ulem is needed to support strikethroughs (\sout)
                                % normalem makes italics be italics, not underlines
    \usepackage{mathrsfs}
    

    
    
    % Colors for the hyperref package
    \definecolor{urlcolor}{rgb}{0,.145,.698}
    \definecolor{linkcolor}{rgb}{.71,0.21,0.01}
    \definecolor{citecolor}{rgb}{.12,.54,.11}

    % ANSI colors
    \definecolor{ansi-black}{HTML}{3E424D}
    \definecolor{ansi-black-intense}{HTML}{282C36}
    \definecolor{ansi-red}{HTML}{E75C58}
    \definecolor{ansi-red-intense}{HTML}{B22B31}
    \definecolor{ansi-green}{HTML}{00A250}
    \definecolor{ansi-green-intense}{HTML}{007427}
    \definecolor{ansi-yellow}{HTML}{DDB62B}
    \definecolor{ansi-yellow-intense}{HTML}{B27D12}
    \definecolor{ansi-blue}{HTML}{208FFB}
    \definecolor{ansi-blue-intense}{HTML}{0065CA}
    \definecolor{ansi-magenta}{HTML}{D160C4}
    \definecolor{ansi-magenta-intense}{HTML}{A03196}
    \definecolor{ansi-cyan}{HTML}{60C6C8}
    \definecolor{ansi-cyan-intense}{HTML}{258F8F}
    \definecolor{ansi-white}{HTML}{C5C1B4}
    \definecolor{ansi-white-intense}{HTML}{A1A6B2}
    \definecolor{ansi-default-inverse-fg}{HTML}{FFFFFF}
    \definecolor{ansi-default-inverse-bg}{HTML}{000000}

    % commands and environments needed by pandoc snippets
    % extracted from the output of `pandoc -s`
    \providecommand{\tightlist}{%
      \setlength{\itemsep}{0pt}\setlength{\parskip}{0pt}}
    \DefineVerbatimEnvironment{Highlighting}{Verbatim}{commandchars=\\\{\}}
    % Add ',fontsize=\small' for more characters per line
    \newenvironment{Shaded}{}{}
    \newcommand{\KeywordTok}[1]{\textcolor[rgb]{0.00,0.44,0.13}{\textbf{{#1}}}}
    \newcommand{\DataTypeTok}[1]{\textcolor[rgb]{0.56,0.13,0.00}{{#1}}}
    \newcommand{\DecValTok}[1]{\textcolor[rgb]{0.25,0.63,0.44}{{#1}}}
    \newcommand{\BaseNTok}[1]{\textcolor[rgb]{0.25,0.63,0.44}{{#1}}}
    \newcommand{\FloatTok}[1]{\textcolor[rgb]{0.25,0.63,0.44}{{#1}}}
    \newcommand{\CharTok}[1]{\textcolor[rgb]{0.25,0.44,0.63}{{#1}}}
    \newcommand{\StringTok}[1]{\textcolor[rgb]{0.25,0.44,0.63}{{#1}}}
    \newcommand{\CommentTok}[1]{\textcolor[rgb]{0.38,0.63,0.69}{\textit{{#1}}}}
    \newcommand{\OtherTok}[1]{\textcolor[rgb]{0.00,0.44,0.13}{{#1}}}
    \newcommand{\AlertTok}[1]{\textcolor[rgb]{1.00,0.00,0.00}{\textbf{{#1}}}}
    \newcommand{\FunctionTok}[1]{\textcolor[rgb]{0.02,0.16,0.49}{{#1}}}
    \newcommand{\RegionMarkerTok}[1]{{#1}}
    \newcommand{\ErrorTok}[1]{\textcolor[rgb]{1.00,0.00,0.00}{\textbf{{#1}}}}
    \newcommand{\NormalTok}[1]{{#1}}
    
    % Additional commands for more recent versions of Pandoc
    \newcommand{\ConstantTok}[1]{\textcolor[rgb]{0.53,0.00,0.00}{{#1}}}
    \newcommand{\SpecialCharTok}[1]{\textcolor[rgb]{0.25,0.44,0.63}{{#1}}}
    \newcommand{\VerbatimStringTok}[1]{\textcolor[rgb]{0.25,0.44,0.63}{{#1}}}
    \newcommand{\SpecialStringTok}[1]{\textcolor[rgb]{0.73,0.40,0.53}{{#1}}}
    \newcommand{\ImportTok}[1]{{#1}}
    \newcommand{\DocumentationTok}[1]{\textcolor[rgb]{0.73,0.13,0.13}{\textit{{#1}}}}
    \newcommand{\AnnotationTok}[1]{\textcolor[rgb]{0.38,0.63,0.69}{\textbf{\textit{{#1}}}}}
    \newcommand{\CommentVarTok}[1]{\textcolor[rgb]{0.38,0.63,0.69}{\textbf{\textit{{#1}}}}}
    \newcommand{\VariableTok}[1]{\textcolor[rgb]{0.10,0.09,0.49}{{#1}}}
    \newcommand{\ControlFlowTok}[1]{\textcolor[rgb]{0.00,0.44,0.13}{\textbf{{#1}}}}
    \newcommand{\OperatorTok}[1]{\textcolor[rgb]{0.40,0.40,0.40}{{#1}}}
    \newcommand{\BuiltInTok}[1]{{#1}}
    \newcommand{\ExtensionTok}[1]{{#1}}
    \newcommand{\PreprocessorTok}[1]{\textcolor[rgb]{0.74,0.48,0.00}{{#1}}}
    \newcommand{\AttributeTok}[1]{\textcolor[rgb]{0.49,0.56,0.16}{{#1}}}
    \newcommand{\InformationTok}[1]{\textcolor[rgb]{0.38,0.63,0.69}{\textbf{\textit{{#1}}}}}
    \newcommand{\WarningTok}[1]{\textcolor[rgb]{0.38,0.63,0.69}{\textbf{\textit{{#1}}}}}
    
    
    % Define a nice break command that doesn't care if a line doesn't already
    % exist.
    \def\br{\hspace*{\fill} \\* }
    % Math Jax compatibility definitions
    \def\gt{>}
    \def\lt{<}
    \let\Oldtex\TeX
    \let\Oldlatex\LaTeX
    \renewcommand{\TeX}{\textrm{\Oldtex}}
    \renewcommand{\LaTeX}{\textrm{\Oldlatex}}
    % Document parameters
    % Document title
    \title{data-wrangling}
    
    
    
    
    

    % Pygments definitions
    
\makeatletter
\def\PY@reset{\let\PY@it=\relax \let\PY@bf=\relax%
    \let\PY@ul=\relax \let\PY@tc=\relax%
    \let\PY@bc=\relax \let\PY@ff=\relax}
\def\PY@tok#1{\csname PY@tok@#1\endcsname}
\def\PY@toks#1+{\ifx\relax#1\empty\else%
    \PY@tok{#1}\expandafter\PY@toks\fi}
\def\PY@do#1{\PY@bc{\PY@tc{\PY@ul{%
    \PY@it{\PY@bf{\PY@ff{#1}}}}}}}
\def\PY#1#2{\PY@reset\PY@toks#1+\relax+\PY@do{#2}}

\expandafter\def\csname PY@tok@w\endcsname{\def\PY@tc##1{\textcolor[rgb]{0.73,0.73,0.73}{##1}}}
\expandafter\def\csname PY@tok@c\endcsname{\let\PY@it=\textit\def\PY@tc##1{\textcolor[rgb]{0.25,0.50,0.50}{##1}}}
\expandafter\def\csname PY@tok@cp\endcsname{\def\PY@tc##1{\textcolor[rgb]{0.74,0.48,0.00}{##1}}}
\expandafter\def\csname PY@tok@k\endcsname{\let\PY@bf=\textbf\def\PY@tc##1{\textcolor[rgb]{0.00,0.50,0.00}{##1}}}
\expandafter\def\csname PY@tok@kp\endcsname{\def\PY@tc##1{\textcolor[rgb]{0.00,0.50,0.00}{##1}}}
\expandafter\def\csname PY@tok@kt\endcsname{\def\PY@tc##1{\textcolor[rgb]{0.69,0.00,0.25}{##1}}}
\expandafter\def\csname PY@tok@o\endcsname{\def\PY@tc##1{\textcolor[rgb]{0.40,0.40,0.40}{##1}}}
\expandafter\def\csname PY@tok@ow\endcsname{\let\PY@bf=\textbf\def\PY@tc##1{\textcolor[rgb]{0.67,0.13,1.00}{##1}}}
\expandafter\def\csname PY@tok@nb\endcsname{\def\PY@tc##1{\textcolor[rgb]{0.00,0.50,0.00}{##1}}}
\expandafter\def\csname PY@tok@nf\endcsname{\def\PY@tc##1{\textcolor[rgb]{0.00,0.00,1.00}{##1}}}
\expandafter\def\csname PY@tok@nc\endcsname{\let\PY@bf=\textbf\def\PY@tc##1{\textcolor[rgb]{0.00,0.00,1.00}{##1}}}
\expandafter\def\csname PY@tok@nn\endcsname{\let\PY@bf=\textbf\def\PY@tc##1{\textcolor[rgb]{0.00,0.00,1.00}{##1}}}
\expandafter\def\csname PY@tok@ne\endcsname{\let\PY@bf=\textbf\def\PY@tc##1{\textcolor[rgb]{0.82,0.25,0.23}{##1}}}
\expandafter\def\csname PY@tok@nv\endcsname{\def\PY@tc##1{\textcolor[rgb]{0.10,0.09,0.49}{##1}}}
\expandafter\def\csname PY@tok@no\endcsname{\def\PY@tc##1{\textcolor[rgb]{0.53,0.00,0.00}{##1}}}
\expandafter\def\csname PY@tok@nl\endcsname{\def\PY@tc##1{\textcolor[rgb]{0.63,0.63,0.00}{##1}}}
\expandafter\def\csname PY@tok@ni\endcsname{\let\PY@bf=\textbf\def\PY@tc##1{\textcolor[rgb]{0.60,0.60,0.60}{##1}}}
\expandafter\def\csname PY@tok@na\endcsname{\def\PY@tc##1{\textcolor[rgb]{0.49,0.56,0.16}{##1}}}
\expandafter\def\csname PY@tok@nt\endcsname{\let\PY@bf=\textbf\def\PY@tc##1{\textcolor[rgb]{0.00,0.50,0.00}{##1}}}
\expandafter\def\csname PY@tok@nd\endcsname{\def\PY@tc##1{\textcolor[rgb]{0.67,0.13,1.00}{##1}}}
\expandafter\def\csname PY@tok@s\endcsname{\def\PY@tc##1{\textcolor[rgb]{0.73,0.13,0.13}{##1}}}
\expandafter\def\csname PY@tok@sd\endcsname{\let\PY@it=\textit\def\PY@tc##1{\textcolor[rgb]{0.73,0.13,0.13}{##1}}}
\expandafter\def\csname PY@tok@si\endcsname{\let\PY@bf=\textbf\def\PY@tc##1{\textcolor[rgb]{0.73,0.40,0.53}{##1}}}
\expandafter\def\csname PY@tok@se\endcsname{\let\PY@bf=\textbf\def\PY@tc##1{\textcolor[rgb]{0.73,0.40,0.13}{##1}}}
\expandafter\def\csname PY@tok@sr\endcsname{\def\PY@tc##1{\textcolor[rgb]{0.73,0.40,0.53}{##1}}}
\expandafter\def\csname PY@tok@ss\endcsname{\def\PY@tc##1{\textcolor[rgb]{0.10,0.09,0.49}{##1}}}
\expandafter\def\csname PY@tok@sx\endcsname{\def\PY@tc##1{\textcolor[rgb]{0.00,0.50,0.00}{##1}}}
\expandafter\def\csname PY@tok@m\endcsname{\def\PY@tc##1{\textcolor[rgb]{0.40,0.40,0.40}{##1}}}
\expandafter\def\csname PY@tok@gh\endcsname{\let\PY@bf=\textbf\def\PY@tc##1{\textcolor[rgb]{0.00,0.00,0.50}{##1}}}
\expandafter\def\csname PY@tok@gu\endcsname{\let\PY@bf=\textbf\def\PY@tc##1{\textcolor[rgb]{0.50,0.00,0.50}{##1}}}
\expandafter\def\csname PY@tok@gd\endcsname{\def\PY@tc##1{\textcolor[rgb]{0.63,0.00,0.00}{##1}}}
\expandafter\def\csname PY@tok@gi\endcsname{\def\PY@tc##1{\textcolor[rgb]{0.00,0.63,0.00}{##1}}}
\expandafter\def\csname PY@tok@gr\endcsname{\def\PY@tc##1{\textcolor[rgb]{1.00,0.00,0.00}{##1}}}
\expandafter\def\csname PY@tok@ge\endcsname{\let\PY@it=\textit}
\expandafter\def\csname PY@tok@gs\endcsname{\let\PY@bf=\textbf}
\expandafter\def\csname PY@tok@gp\endcsname{\let\PY@bf=\textbf\def\PY@tc##1{\textcolor[rgb]{0.00,0.00,0.50}{##1}}}
\expandafter\def\csname PY@tok@go\endcsname{\def\PY@tc##1{\textcolor[rgb]{0.53,0.53,0.53}{##1}}}
\expandafter\def\csname PY@tok@gt\endcsname{\def\PY@tc##1{\textcolor[rgb]{0.00,0.27,0.87}{##1}}}
\expandafter\def\csname PY@tok@err\endcsname{\def\PY@bc##1{\setlength{\fboxsep}{0pt}\fcolorbox[rgb]{1.00,0.00,0.00}{1,1,1}{\strut ##1}}}
\expandafter\def\csname PY@tok@kc\endcsname{\let\PY@bf=\textbf\def\PY@tc##1{\textcolor[rgb]{0.00,0.50,0.00}{##1}}}
\expandafter\def\csname PY@tok@kd\endcsname{\let\PY@bf=\textbf\def\PY@tc##1{\textcolor[rgb]{0.00,0.50,0.00}{##1}}}
\expandafter\def\csname PY@tok@kn\endcsname{\let\PY@bf=\textbf\def\PY@tc##1{\textcolor[rgb]{0.00,0.50,0.00}{##1}}}
\expandafter\def\csname PY@tok@kr\endcsname{\let\PY@bf=\textbf\def\PY@tc##1{\textcolor[rgb]{0.00,0.50,0.00}{##1}}}
\expandafter\def\csname PY@tok@bp\endcsname{\def\PY@tc##1{\textcolor[rgb]{0.00,0.50,0.00}{##1}}}
\expandafter\def\csname PY@tok@fm\endcsname{\def\PY@tc##1{\textcolor[rgb]{0.00,0.00,1.00}{##1}}}
\expandafter\def\csname PY@tok@vc\endcsname{\def\PY@tc##1{\textcolor[rgb]{0.10,0.09,0.49}{##1}}}
\expandafter\def\csname PY@tok@vg\endcsname{\def\PY@tc##1{\textcolor[rgb]{0.10,0.09,0.49}{##1}}}
\expandafter\def\csname PY@tok@vi\endcsname{\def\PY@tc##1{\textcolor[rgb]{0.10,0.09,0.49}{##1}}}
\expandafter\def\csname PY@tok@vm\endcsname{\def\PY@tc##1{\textcolor[rgb]{0.10,0.09,0.49}{##1}}}
\expandafter\def\csname PY@tok@sa\endcsname{\def\PY@tc##1{\textcolor[rgb]{0.73,0.13,0.13}{##1}}}
\expandafter\def\csname PY@tok@sb\endcsname{\def\PY@tc##1{\textcolor[rgb]{0.73,0.13,0.13}{##1}}}
\expandafter\def\csname PY@tok@sc\endcsname{\def\PY@tc##1{\textcolor[rgb]{0.73,0.13,0.13}{##1}}}
\expandafter\def\csname PY@tok@dl\endcsname{\def\PY@tc##1{\textcolor[rgb]{0.73,0.13,0.13}{##1}}}
\expandafter\def\csname PY@tok@s2\endcsname{\def\PY@tc##1{\textcolor[rgb]{0.73,0.13,0.13}{##1}}}
\expandafter\def\csname PY@tok@sh\endcsname{\def\PY@tc##1{\textcolor[rgb]{0.73,0.13,0.13}{##1}}}
\expandafter\def\csname PY@tok@s1\endcsname{\def\PY@tc##1{\textcolor[rgb]{0.73,0.13,0.13}{##1}}}
\expandafter\def\csname PY@tok@mb\endcsname{\def\PY@tc##1{\textcolor[rgb]{0.40,0.40,0.40}{##1}}}
\expandafter\def\csname PY@tok@mf\endcsname{\def\PY@tc##1{\textcolor[rgb]{0.40,0.40,0.40}{##1}}}
\expandafter\def\csname PY@tok@mh\endcsname{\def\PY@tc##1{\textcolor[rgb]{0.40,0.40,0.40}{##1}}}
\expandafter\def\csname PY@tok@mi\endcsname{\def\PY@tc##1{\textcolor[rgb]{0.40,0.40,0.40}{##1}}}
\expandafter\def\csname PY@tok@il\endcsname{\def\PY@tc##1{\textcolor[rgb]{0.40,0.40,0.40}{##1}}}
\expandafter\def\csname PY@tok@mo\endcsname{\def\PY@tc##1{\textcolor[rgb]{0.40,0.40,0.40}{##1}}}
\expandafter\def\csname PY@tok@ch\endcsname{\let\PY@it=\textit\def\PY@tc##1{\textcolor[rgb]{0.25,0.50,0.50}{##1}}}
\expandafter\def\csname PY@tok@cm\endcsname{\let\PY@it=\textit\def\PY@tc##1{\textcolor[rgb]{0.25,0.50,0.50}{##1}}}
\expandafter\def\csname PY@tok@cpf\endcsname{\let\PY@it=\textit\def\PY@tc##1{\textcolor[rgb]{0.25,0.50,0.50}{##1}}}
\expandafter\def\csname PY@tok@c1\endcsname{\let\PY@it=\textit\def\PY@tc##1{\textcolor[rgb]{0.25,0.50,0.50}{##1}}}
\expandafter\def\csname PY@tok@cs\endcsname{\let\PY@it=\textit\def\PY@tc##1{\textcolor[rgb]{0.25,0.50,0.50}{##1}}}

\def\PYZbs{\char`\\}
\def\PYZus{\char`\_}
\def\PYZob{\char`\{}
\def\PYZcb{\char`\}}
\def\PYZca{\char`\^}
\def\PYZam{\char`\&}
\def\PYZlt{\char`\<}
\def\PYZgt{\char`\>}
\def\PYZsh{\char`\#}
\def\PYZpc{\char`\%}
\def\PYZdl{\char`\$}
\def\PYZhy{\char`\-}
\def\PYZsq{\char`\'}
\def\PYZdq{\char`\"}
\def\PYZti{\char`\~}
% for compatibility with earlier versions
\def\PYZat{@}
\def\PYZlb{[}
\def\PYZrb{]}
\makeatother


    % Exact colors from NB
    \definecolor{incolor}{rgb}{0.0, 0.0, 0.5}
    \definecolor{outcolor}{rgb}{0.545, 0.0, 0.0}



    
    % Prevent overflowing lines due to hard-to-break entities
    \sloppy 
    % Setup hyperref package
    \hypersetup{
      breaklinks=true,  % so long urls are correctly broken across lines
      colorlinks=true,
      urlcolor=urlcolor,
      linkcolor=linkcolor,
      citecolor=citecolor,
      }
    % Slightly bigger margins than the latex defaults
    
    \geometry{verbose,tmargin=1in,bmargin=1in,lmargin=1in,rmargin=1in}
    
    

    \begin{document}
    
    
    \maketitle
    
    

    
    \begin{verbatim}
<a href="https://cocl.us/corsera_da0101en_notebook_top">
     <img src="https://s3-api.us-geo.objectstorage.softlayer.net/cf-courses-data/CognitiveClass/DA0101EN/Images/TopAd.png" width="750" align="center">
</a>
\end{verbatim}

    Data Analysis with Python

    Data Wrangling

    Welcome!

By the end of this notebook, you will have learned the basics of Data
Wrangling!

    Table of content

\begin{verbatim}
<li><a href="#identify_handle_missing_values">Identify and handle missing values</a>
    <ul>
        <li><a href="#identify_missing_values">Identify missing values</a></li>
        <li><a href="#deal_missing_values">Deal with missing values</a></li>
        <li><a href="#correct_data_format">Correct data format</a></li>
    </ul>
</li>
<li><a href="#data_standardization">Data standardization</a></li>
<li><a href="#data_normalization">Data Normalization (centering/scaling)</a></li>
<li><a href="#binning">Binning</a></li>
<li><a href="#indicator">Indicator variable</a></li>
\end{verbatim}

Estimated Time Needed: 30 min

    What is the purpose of Data Wrangling?

    Data Wrangling is the process of converting data from the initial format
to a format that may be better for analysis.

    What is the fuel consumption (L/100k) rate for the diesel car?

    Import data

You can find the "Automobile Data Set" from the following link:
https://archive.ics.uci.edu/ml/machine-learning-databases/autos/imports-85.data.
We will be using this data set throughout this course.

    Import pandas

    \begin{Verbatim}[commandchars=\\\{\}]
{\color{incolor}In [{\color{incolor}1}]:} \PY{k+kn}{import} \PY{n+nn}{pandas} \PY{k}{as} \PY{n+nn}{pd}
        \PY{k+kn}{import} \PY{n+nn}{matplotlib}\PY{n+nn}{.}\PY{n+nn}{pylab} \PY{k}{as} \PY{n+nn}{plt}
\end{Verbatim}

    Reading the data set from the URL and adding the related headers.

    URL of the dataset

    This dataset was hosted on IBM Cloud object click HERE for free storage

    \begin{Verbatim}[commandchars=\\\{\}]
{\color{incolor}In [{\color{incolor}2}]:} \PY{n}{filename} \PY{o}{=} \PY{l+s+s2}{\PYZdq{}}\PY{l+s+s2}{https://s3\PYZhy{}api.us\PYZhy{}geo.objectstorage.softlayer.net/cf\PYZhy{}courses\PYZhy{}data/CognitiveClass/DA0101EN/auto.csv}\PY{l+s+s2}{\PYZdq{}}
\end{Verbatim}

    Python list headers containing name of headers

    \begin{Verbatim}[commandchars=\\\{\}]
{\color{incolor}In [{\color{incolor}3}]:} \PY{n}{headers} \PY{o}{=} \PY{p}{[}\PY{l+s+s2}{\PYZdq{}}\PY{l+s+s2}{symboling}\PY{l+s+s2}{\PYZdq{}}\PY{p}{,}\PY{l+s+s2}{\PYZdq{}}\PY{l+s+s2}{normalized\PYZhy{}losses}\PY{l+s+s2}{\PYZdq{}}\PY{p}{,}\PY{l+s+s2}{\PYZdq{}}\PY{l+s+s2}{make}\PY{l+s+s2}{\PYZdq{}}\PY{p}{,}\PY{l+s+s2}{\PYZdq{}}\PY{l+s+s2}{fuel\PYZhy{}type}\PY{l+s+s2}{\PYZdq{}}\PY{p}{,}\PY{l+s+s2}{\PYZdq{}}\PY{l+s+s2}{aspiration}\PY{l+s+s2}{\PYZdq{}}\PY{p}{,} \PY{l+s+s2}{\PYZdq{}}\PY{l+s+s2}{num\PYZhy{}of\PYZhy{}doors}\PY{l+s+s2}{\PYZdq{}}\PY{p}{,}\PY{l+s+s2}{\PYZdq{}}\PY{l+s+s2}{body\PYZhy{}style}\PY{l+s+s2}{\PYZdq{}}\PY{p}{,}
                 \PY{l+s+s2}{\PYZdq{}}\PY{l+s+s2}{drive\PYZhy{}wheels}\PY{l+s+s2}{\PYZdq{}}\PY{p}{,}\PY{l+s+s2}{\PYZdq{}}\PY{l+s+s2}{engine\PYZhy{}location}\PY{l+s+s2}{\PYZdq{}}\PY{p}{,}\PY{l+s+s2}{\PYZdq{}}\PY{l+s+s2}{wheel\PYZhy{}base}\PY{l+s+s2}{\PYZdq{}}\PY{p}{,} \PY{l+s+s2}{\PYZdq{}}\PY{l+s+s2}{length}\PY{l+s+s2}{\PYZdq{}}\PY{p}{,}\PY{l+s+s2}{\PYZdq{}}\PY{l+s+s2}{width}\PY{l+s+s2}{\PYZdq{}}\PY{p}{,}\PY{l+s+s2}{\PYZdq{}}\PY{l+s+s2}{height}\PY{l+s+s2}{\PYZdq{}}\PY{p}{,}\PY{l+s+s2}{\PYZdq{}}\PY{l+s+s2}{curb\PYZhy{}weight}\PY{l+s+s2}{\PYZdq{}}\PY{p}{,}\PY{l+s+s2}{\PYZdq{}}\PY{l+s+s2}{engine\PYZhy{}type}\PY{l+s+s2}{\PYZdq{}}\PY{p}{,}
                 \PY{l+s+s2}{\PYZdq{}}\PY{l+s+s2}{num\PYZhy{}of\PYZhy{}cylinders}\PY{l+s+s2}{\PYZdq{}}\PY{p}{,} \PY{l+s+s2}{\PYZdq{}}\PY{l+s+s2}{engine\PYZhy{}size}\PY{l+s+s2}{\PYZdq{}}\PY{p}{,}\PY{l+s+s2}{\PYZdq{}}\PY{l+s+s2}{fuel\PYZhy{}system}\PY{l+s+s2}{\PYZdq{}}\PY{p}{,}\PY{l+s+s2}{\PYZdq{}}\PY{l+s+s2}{bore}\PY{l+s+s2}{\PYZdq{}}\PY{p}{,}\PY{l+s+s2}{\PYZdq{}}\PY{l+s+s2}{stroke}\PY{l+s+s2}{\PYZdq{}}\PY{p}{,}\PY{l+s+s2}{\PYZdq{}}\PY{l+s+s2}{compression\PYZhy{}ratio}\PY{l+s+s2}{\PYZdq{}}\PY{p}{,}\PY{l+s+s2}{\PYZdq{}}\PY{l+s+s2}{horsepower}\PY{l+s+s2}{\PYZdq{}}\PY{p}{,}
                 \PY{l+s+s2}{\PYZdq{}}\PY{l+s+s2}{peak\PYZhy{}rpm}\PY{l+s+s2}{\PYZdq{}}\PY{p}{,}\PY{l+s+s2}{\PYZdq{}}\PY{l+s+s2}{city\PYZhy{}mpg}\PY{l+s+s2}{\PYZdq{}}\PY{p}{,}\PY{l+s+s2}{\PYZdq{}}\PY{l+s+s2}{highway\PYZhy{}mpg}\PY{l+s+s2}{\PYZdq{}}\PY{p}{,}\PY{l+s+s2}{\PYZdq{}}\PY{l+s+s2}{price}\PY{l+s+s2}{\PYZdq{}}\PY{p}{]}
\end{Verbatim}

    Use the Pandas method read\_csv() to load the data from the web address.
Set the parameter "names" equal to the Python list "headers".

    \begin{Verbatim}[commandchars=\\\{\}]
{\color{incolor}In [{\color{incolor}4}]:} \PY{n}{df} \PY{o}{=} \PY{n}{pd}\PY{o}{.}\PY{n}{read\PYZus{}csv}\PY{p}{(}\PY{n}{filename}\PY{p}{,} \PY{n}{names} \PY{o}{=} \PY{n}{headers}\PY{p}{)}
\end{Verbatim}

    Use the method head() to display the first five rows of the dataframe.

    \begin{Verbatim}[commandchars=\\\{\}]
{\color{incolor}In [{\color{incolor}5}]:} \PY{c+c1}{\PYZsh{} To see what the data set looks like, we\PYZsq{}ll use the head() method.}
        \PY{n}{df}\PY{o}{.}\PY{n}{head}\PY{p}{(}\PY{p}{)}
\end{Verbatim}

\begin{Verbatim}[commandchars=\\\{\}]
{\color{outcolor}Out[{\color{outcolor}5}]:}    symboling normalized-losses         make fuel-type aspiration num-of-doors  \textbackslash{}
        0          3                 ?  alfa-romero       gas        std          two   
        1          3                 ?  alfa-romero       gas        std          two   
        2          1                 ?  alfa-romero       gas        std          two   
        3          2               164         audi       gas        std         four   
        4          2               164         audi       gas        std         four   
        
            body-style drive-wheels engine-location  wheel-base  {\ldots}  engine-size  \textbackslash{}
        0  convertible          rwd           front        88.6  {\ldots}          130   
        1  convertible          rwd           front        88.6  {\ldots}          130   
        2    hatchback          rwd           front        94.5  {\ldots}          152   
        3        sedan          fwd           front        99.8  {\ldots}          109   
        4        sedan          4wd           front        99.4  {\ldots}          136   
        
           fuel-system  bore  stroke compression-ratio horsepower  peak-rpm city-mpg  \textbackslash{}
        0         mpfi  3.47    2.68               9.0        111      5000       21   
        1         mpfi  3.47    2.68               9.0        111      5000       21   
        2         mpfi  2.68    3.47               9.0        154      5000       19   
        3         mpfi  3.19    3.40              10.0        102      5500       24   
        4         mpfi  3.19    3.40               8.0        115      5500       18   
        
          highway-mpg  price  
        0          27  13495  
        1          27  16500  
        2          26  16500  
        3          30  13950  
        4          22  17450  
        
        [5 rows x 26 columns]
\end{Verbatim}
            
    As we can see, several question marks appeared in the dataframe; those
are missing values which may hinder our further analysis.

So, how do we identify all those missing values and deal with them?

How to work with missing data?

Steps for working with missing data:

\begin{verbatim}
<li>dentify missing data</li>
<li>deal with missing data</li>
<li>correct data format</li>
\end{verbatim}

    Identify and handle missing values

Identify missing values

Convert "?" to NaN

In the car dataset, missing data comes with the question mark "?". We
replace "?" with NaN (Not a Number), which is Python's default missing
value marker, for reasons of computational speed and convenience. Here
we use the function:

to replace A by B

    \begin{Verbatim}[commandchars=\\\{\}]
{\color{incolor}In [{\color{incolor}6}]:} \PY{k+kn}{import} \PY{n+nn}{numpy} \PY{k}{as} \PY{n+nn}{np}
        
        \PY{c+c1}{\PYZsh{} replace \PYZdq{}?\PYZdq{} to NaN}
        \PY{n}{df}\PY{o}{.}\PY{n}{replace}\PY{p}{(}\PY{l+s+s2}{\PYZdq{}}\PY{l+s+s2}{?}\PY{l+s+s2}{\PYZdq{}}\PY{p}{,} \PY{n}{np}\PY{o}{.}\PY{n}{nan}\PY{p}{,} \PY{n}{inplace} \PY{o}{=} \PY{k+kc}{True}\PY{p}{)}
        \PY{n}{df}\PY{o}{.}\PY{n}{head}\PY{p}{(}\PY{l+m+mi}{5}\PY{p}{)}
\end{Verbatim}

\begin{Verbatim}[commandchars=\\\{\}]
{\color{outcolor}Out[{\color{outcolor}6}]:}    symboling normalized-losses         make fuel-type aspiration num-of-doors  \textbackslash{}
        0          3               NaN  alfa-romero       gas        std          two   
        1          3               NaN  alfa-romero       gas        std          two   
        2          1               NaN  alfa-romero       gas        std          two   
        3          2               164         audi       gas        std         four   
        4          2               164         audi       gas        std         four   
        
            body-style drive-wheels engine-location  wheel-base  {\ldots}  engine-size  \textbackslash{}
        0  convertible          rwd           front        88.6  {\ldots}          130   
        1  convertible          rwd           front        88.6  {\ldots}          130   
        2    hatchback          rwd           front        94.5  {\ldots}          152   
        3        sedan          fwd           front        99.8  {\ldots}          109   
        4        sedan          4wd           front        99.4  {\ldots}          136   
        
           fuel-system  bore  stroke compression-ratio horsepower  peak-rpm city-mpg  \textbackslash{}
        0         mpfi  3.47    2.68               9.0        111      5000       21   
        1         mpfi  3.47    2.68               9.0        111      5000       21   
        2         mpfi  2.68    3.47               9.0        154      5000       19   
        3         mpfi  3.19    3.40              10.0        102      5500       24   
        4         mpfi  3.19    3.40               8.0        115      5500       18   
        
          highway-mpg  price  
        0          27  13495  
        1          27  16500  
        2          26  16500  
        3          30  13950  
        4          22  17450  
        
        [5 rows x 26 columns]
\end{Verbatim}
            
    dentify\_missing\_values

Evaluating for Missing Data

The missing values are converted to Python's default. We use Python's
built-in functions to identify these missing values. There are two
methods to detect missing data:

\begin{verbatim}
<li><b>.isnull()</b></li>
<li><b>.notnull()</b></li>
\end{verbatim}

The output is a boolean value indicating whether the value that is
passed into the argument is in fact missing data.

    \begin{Verbatim}[commandchars=\\\{\}]
{\color{incolor}In [{\color{incolor}7}]:} \PY{n}{missing\PYZus{}data} \PY{o}{=} \PY{n}{df}\PY{o}{.}\PY{n}{isnull}\PY{p}{(}\PY{p}{)}
        \PY{n}{missing\PYZus{}data}\PY{o}{.}\PY{n}{head}\PY{p}{(}\PY{l+m+mi}{5}\PY{p}{)}
\end{Verbatim}

\begin{Verbatim}[commandchars=\\\{\}]
{\color{outcolor}Out[{\color{outcolor}7}]:}    symboling  normalized-losses   make  fuel-type  aspiration  num-of-doors  \textbackslash{}
        0      False               True  False      False       False         False   
        1      False               True  False      False       False         False   
        2      False               True  False      False       False         False   
        3      False              False  False      False       False         False   
        4      False              False  False      False       False         False   
        
           body-style  drive-wheels  engine-location  wheel-base  {\ldots}  engine-size  \textbackslash{}
        0       False         False            False       False  {\ldots}        False   
        1       False         False            False       False  {\ldots}        False   
        2       False         False            False       False  {\ldots}        False   
        3       False         False            False       False  {\ldots}        False   
        4       False         False            False       False  {\ldots}        False   
        
           fuel-system   bore  stroke  compression-ratio  horsepower  peak-rpm  \textbackslash{}
        0        False  False   False              False       False     False   
        1        False  False   False              False       False     False   
        2        False  False   False              False       False     False   
        3        False  False   False              False       False     False   
        4        False  False   False              False       False     False   
        
           city-mpg  highway-mpg  price  
        0     False        False  False  
        1     False        False  False  
        2     False        False  False  
        3     False        False  False  
        4     False        False  False  
        
        [5 rows x 26 columns]
\end{Verbatim}
            
    "True" stands for missing value, while "False" stands for not missing
value.

    Count missing values in each column

Using a for loop in Python, we can quickly figure out the number of
missing values in each column. As mentioned above, "True" represents a
missing value, "False" means the value is present in the dataset. In the
body of the for loop the method ".value\_counts()" counts the number of
"True" values.

    \begin{Verbatim}[commandchars=\\\{\}]
{\color{incolor}In [{\color{incolor}8}]:} \PY{k}{for} \PY{n}{column} \PY{o+ow}{in} \PY{n}{missing\PYZus{}data}\PY{o}{.}\PY{n}{columns}\PY{o}{.}\PY{n}{values}\PY{o}{.}\PY{n}{tolist}\PY{p}{(}\PY{p}{)}\PY{p}{:}
            \PY{n+nb}{print}\PY{p}{(}\PY{n}{column}\PY{p}{)}
            \PY{n+nb}{print} \PY{p}{(}\PY{n}{missing\PYZus{}data}\PY{p}{[}\PY{n}{column}\PY{p}{]}\PY{o}{.}\PY{n}{value\PYZus{}counts}\PY{p}{(}\PY{p}{)}\PY{p}{)}
            \PY{n+nb}{print}\PY{p}{(}\PY{l+s+s2}{\PYZdq{}}\PY{l+s+s2}{\PYZdq{}}\PY{p}{)}    
\end{Verbatim}

    \begin{Verbatim}[commandchars=\\\{\}]
symboling
False    205
Name: symboling, dtype: int64

normalized-losses
False    164
True      41
Name: normalized-losses, dtype: int64

make
False    205
Name: make, dtype: int64

fuel-type
False    205
Name: fuel-type, dtype: int64

aspiration
False    205
Name: aspiration, dtype: int64

num-of-doors
False    203
True       2
Name: num-of-doors, dtype: int64

body-style
False    205
Name: body-style, dtype: int64

drive-wheels
False    205
Name: drive-wheels, dtype: int64

engine-location
False    205
Name: engine-location, dtype: int64

wheel-base
False    205
Name: wheel-base, dtype: int64

length
False    205
Name: length, dtype: int64

width
False    205
Name: width, dtype: int64

height
False    205
Name: height, dtype: int64

curb-weight
False    205
Name: curb-weight, dtype: int64

engine-type
False    205
Name: engine-type, dtype: int64

num-of-cylinders
False    205
Name: num-of-cylinders, dtype: int64

engine-size
False    205
Name: engine-size, dtype: int64

fuel-system
False    205
Name: fuel-system, dtype: int64

bore
False    201
True       4
Name: bore, dtype: int64

stroke
False    201
True       4
Name: stroke, dtype: int64

compression-ratio
False    205
Name: compression-ratio, dtype: int64

horsepower
False    203
True       2
Name: horsepower, dtype: int64

peak-rpm
False    203
True       2
Name: peak-rpm, dtype: int64

city-mpg
False    205
Name: city-mpg, dtype: int64

highway-mpg
False    205
Name: highway-mpg, dtype: int64

price
False    201
True       4
Name: price, dtype: int64


    \end{Verbatim}

    Based on the summary above, each column has 205 rows of data, seven
columns containing missing data:

\begin{verbatim}
<li>"normalized-losses": 41 missing data</li>
<li>"num-of-doors": 2 missing data</li>
<li>"bore": 4 missing data</li>
<li>"stroke" : 4 missing data</li>
<li>"horsepower": 2 missing data</li>
<li>"peak-rpm": 2 missing data</li>
<li>"price": 4 missing data</li>
\end{verbatim}

    Deal with missing data

How to deal with missing data?

\begin{verbatim}
<li>drop data<br>
    a. drop the whole row<br>
    b. drop the whole column
</li>
<li>replace data<br>
    a. replace it by mean<br>
    b. replace it by frequency<br>
    c. replace it based on other functions
</li>
\end{verbatim}

    Whole columns should be dropped only if most entries in the column are
empty. In our dataset, none of the columns are empty enough to drop
entirely. We have some freedom in choosing which method to replace data;
however, some methods may seem more reasonable than others. We will
apply each method to many different columns:

Replace by mean:

\begin{verbatim}
<li>"normalized-losses": 41 missing data, replace them with mean</li>
<li>"stroke": 4 missing data, replace them with mean</li>
<li>"bore": 4 missing data, replace them with mean</li>
<li>"horsepower": 2 missing data, replace them with mean</li>
<li>"peak-rpm": 2 missing data, replace them with mean</li>
\end{verbatim}

Replace by frequency:

\begin{verbatim}
<li>"num-of-doors": 2 missing data, replace them with "four". 
    <ul>
        <li>Reason: 84% sedans is four doors. Since four doors is most frequent, it is most likely to occur</li>
    </ul>
</li>
\end{verbatim}

Drop the whole row:

\begin{verbatim}
<li>"price": 4 missing data, simply delete the whole row
    <ul>
        <li>Reason: price is what we want to predict. Any data entry without price data cannot be used for prediction; therefore any row now without price data is not useful to us</li>
    </ul>
</li>
\end{verbatim}

    Calculate the average of the column

    \begin{Verbatim}[commandchars=\\\{\}]
{\color{incolor}In [{\color{incolor}9}]:} \PY{n}{avg\PYZus{}norm\PYZus{}loss} \PY{o}{=} \PY{n}{df}\PY{p}{[}\PY{l+s+s2}{\PYZdq{}}\PY{l+s+s2}{normalized\PYZhy{}losses}\PY{l+s+s2}{\PYZdq{}}\PY{p}{]}\PY{o}{.}\PY{n}{astype}\PY{p}{(}\PY{l+s+s2}{\PYZdq{}}\PY{l+s+s2}{float}\PY{l+s+s2}{\PYZdq{}}\PY{p}{)}\PY{o}{.}\PY{n}{mean}\PY{p}{(}\PY{n}{axis}\PY{o}{=}\PY{l+m+mi}{0}\PY{p}{)}
        \PY{n+nb}{print}\PY{p}{(}\PY{l+s+s2}{\PYZdq{}}\PY{l+s+s2}{Average of normalized\PYZhy{}losses:}\PY{l+s+s2}{\PYZdq{}}\PY{p}{,} \PY{n}{avg\PYZus{}norm\PYZus{}loss}\PY{p}{)}
\end{Verbatim}

    \begin{Verbatim}[commandchars=\\\{\}]
Average of normalized-losses: 122.0

    \end{Verbatim}

    Replace "NaN" by mean value in "normalized-losses" column

    \begin{Verbatim}[commandchars=\\\{\}]
{\color{incolor}In [{\color{incolor}11}]:} \PY{n}{df}\PY{p}{[}\PY{l+s+s2}{\PYZdq{}}\PY{l+s+s2}{normalized\PYZhy{}losses}\PY{l+s+s2}{\PYZdq{}}\PY{p}{]}\PY{o}{.}\PY{n}{replace}\PY{p}{(}\PY{n}{np}\PY{o}{.}\PY{n}{nan}\PY{p}{,} \PY{n}{avg\PYZus{}norm\PYZus{}loss}\PY{p}{,} \PY{n}{inplace}\PY{o}{=}\PY{k+kc}{True}\PY{p}{)}
\end{Verbatim}

    Calculate the mean value for 'bore' column

    \begin{Verbatim}[commandchars=\\\{\}]
{\color{incolor}In [{\color{incolor}12}]:} \PY{n}{avg\PYZus{}bore}\PY{o}{=}\PY{n}{df}\PY{p}{[}\PY{l+s+s1}{\PYZsq{}}\PY{l+s+s1}{bore}\PY{l+s+s1}{\PYZsq{}}\PY{p}{]}\PY{o}{.}\PY{n}{astype}\PY{p}{(}\PY{l+s+s1}{\PYZsq{}}\PY{l+s+s1}{float}\PY{l+s+s1}{\PYZsq{}}\PY{p}{)}\PY{o}{.}\PY{n}{mean}\PY{p}{(}\PY{n}{axis}\PY{o}{=}\PY{l+m+mi}{0}\PY{p}{)}
         \PY{n+nb}{print}\PY{p}{(}\PY{l+s+s2}{\PYZdq{}}\PY{l+s+s2}{Average of bore:}\PY{l+s+s2}{\PYZdq{}}\PY{p}{,} \PY{n}{avg\PYZus{}bore}\PY{p}{)}
\end{Verbatim}

    \begin{Verbatim}[commandchars=\\\{\}]
Average of bore: 3.3297512437810943

    \end{Verbatim}

    Replace NaN by mean value

    \begin{Verbatim}[commandchars=\\\{\}]
{\color{incolor}In [{\color{incolor}13}]:} \PY{n}{df}\PY{p}{[}\PY{l+s+s2}{\PYZdq{}}\PY{l+s+s2}{bore}\PY{l+s+s2}{\PYZdq{}}\PY{p}{]}\PY{o}{.}\PY{n}{replace}\PY{p}{(}\PY{n}{np}\PY{o}{.}\PY{n}{nan}\PY{p}{,} \PY{n}{avg\PYZus{}bore}\PY{p}{,} \PY{n}{inplace}\PY{o}{=}\PY{k+kc}{True}\PY{p}{)}
\end{Verbatim}

    Question \#1:

According to the example above, replace NaN in "stroke" column by mean.

    \begin{Verbatim}[commandchars=\\\{\}]
{\color{incolor}In [{\color{incolor} }]:} \PY{c+c1}{\PYZsh{} Write your code below and press Shift+Enter to execute }
\end{Verbatim}

    Double-click here for the solution.

    Calculate the mean value for the 'horsepower' column:

    \begin{Verbatim}[commandchars=\\\{\}]
{\color{incolor}In [{\color{incolor} }]:} \PY{n}{avg\PYZus{}horsepower} \PY{o}{=} \PY{n}{df}\PY{p}{[}\PY{l+s+s1}{\PYZsq{}}\PY{l+s+s1}{horsepower}\PY{l+s+s1}{\PYZsq{}}\PY{p}{]}\PY{o}{.}\PY{n}{astype}\PY{p}{(}\PY{l+s+s1}{\PYZsq{}}\PY{l+s+s1}{float}\PY{l+s+s1}{\PYZsq{}}\PY{p}{)}\PY{o}{.}\PY{n}{mean}\PY{p}{(}\PY{n}{axis}\PY{o}{=}\PY{l+m+mi}{0}\PY{p}{)}
        \PY{n+nb}{print}\PY{p}{(}\PY{l+s+s2}{\PYZdq{}}\PY{l+s+s2}{Average horsepower:}\PY{l+s+s2}{\PYZdq{}}\PY{p}{,} \PY{n}{avg\PYZus{}horsepower}\PY{p}{)}
\end{Verbatim}

    Replace "NaN" by mean value:

    \begin{Verbatim}[commandchars=\\\{\}]
{\color{incolor}In [{\color{incolor} }]:} \PY{n}{df}\PY{p}{[}\PY{l+s+s1}{\PYZsq{}}\PY{l+s+s1}{horsepower}\PY{l+s+s1}{\PYZsq{}}\PY{p}{]}\PY{o}{.}\PY{n}{replace}\PY{p}{(}\PY{n}{np}\PY{o}{.}\PY{n}{nan}\PY{p}{,} \PY{n}{avg\PYZus{}horsepower}\PY{p}{,} \PY{n}{inplace}\PY{o}{=}\PY{k+kc}{True}\PY{p}{)}
\end{Verbatim}

    Calculate the mean value for 'peak-rpm' column:

    \begin{Verbatim}[commandchars=\\\{\}]
{\color{incolor}In [{\color{incolor} }]:} \PY{n}{avg\PYZus{}peakrpm}\PY{o}{=}\PY{n}{df}\PY{p}{[}\PY{l+s+s1}{\PYZsq{}}\PY{l+s+s1}{peak\PYZhy{}rpm}\PY{l+s+s1}{\PYZsq{}}\PY{p}{]}\PY{o}{.}\PY{n}{astype}\PY{p}{(}\PY{l+s+s1}{\PYZsq{}}\PY{l+s+s1}{float}\PY{l+s+s1}{\PYZsq{}}\PY{p}{)}\PY{o}{.}\PY{n}{mean}\PY{p}{(}\PY{n}{axis}\PY{o}{=}\PY{l+m+mi}{0}\PY{p}{)}
        \PY{n+nb}{print}\PY{p}{(}\PY{l+s+s2}{\PYZdq{}}\PY{l+s+s2}{Average peak rpm:}\PY{l+s+s2}{\PYZdq{}}\PY{p}{,} \PY{n}{avg\PYZus{}peakrpm}\PY{p}{)}
\end{Verbatim}

    Replace NaN by mean value:

    \begin{Verbatim}[commandchars=\\\{\}]
{\color{incolor}In [{\color{incolor} }]:} \PY{n}{df}\PY{p}{[}\PY{l+s+s1}{\PYZsq{}}\PY{l+s+s1}{peak\PYZhy{}rpm}\PY{l+s+s1}{\PYZsq{}}\PY{p}{]}\PY{o}{.}\PY{n}{replace}\PY{p}{(}\PY{n}{np}\PY{o}{.}\PY{n}{nan}\PY{p}{,} \PY{n}{avg\PYZus{}peakrpm}\PY{p}{,} \PY{n}{inplace}\PY{o}{=}\PY{k+kc}{True}\PY{p}{)}
\end{Verbatim}

    To see which values are present in a particular column, we can use the
".value\_counts()" method:

    \begin{Verbatim}[commandchars=\\\{\}]
{\color{incolor}In [{\color{incolor} }]:} \PY{n}{df}\PY{p}{[}\PY{l+s+s1}{\PYZsq{}}\PY{l+s+s1}{num\PYZhy{}of\PYZhy{}doors}\PY{l+s+s1}{\PYZsq{}}\PY{p}{]}\PY{o}{.}\PY{n}{value\PYZus{}counts}\PY{p}{(}\PY{p}{)}
\end{Verbatim}

    We can see that four doors are the most common type. We can also use the
".idxmax()" method to calculate for us the most common type
automatically:

    \begin{Verbatim}[commandchars=\\\{\}]
{\color{incolor}In [{\color{incolor} }]:} \PY{n}{df}\PY{p}{[}\PY{l+s+s1}{\PYZsq{}}\PY{l+s+s1}{num\PYZhy{}of\PYZhy{}doors}\PY{l+s+s1}{\PYZsq{}}\PY{p}{]}\PY{o}{.}\PY{n}{value\PYZus{}counts}\PY{p}{(}\PY{p}{)}\PY{o}{.}\PY{n}{idxmax}\PY{p}{(}\PY{p}{)}
\end{Verbatim}

    The replacement procedure is very similar to what we have seen
previously

    \begin{Verbatim}[commandchars=\\\{\}]
{\color{incolor}In [{\color{incolor} }]:} \PY{c+c1}{\PYZsh{}replace the missing \PYZsq{}num\PYZhy{}of\PYZhy{}doors\PYZsq{} values by the most frequent }
        \PY{n}{df}\PY{p}{[}\PY{l+s+s2}{\PYZdq{}}\PY{l+s+s2}{num\PYZhy{}of\PYZhy{}doors}\PY{l+s+s2}{\PYZdq{}}\PY{p}{]}\PY{o}{.}\PY{n}{replace}\PY{p}{(}\PY{n}{np}\PY{o}{.}\PY{n}{nan}\PY{p}{,} \PY{l+s+s2}{\PYZdq{}}\PY{l+s+s2}{four}\PY{l+s+s2}{\PYZdq{}}\PY{p}{,} \PY{n}{inplace}\PY{o}{=}\PY{k+kc}{True}\PY{p}{)}
\end{Verbatim}

    Finally, let's drop all rows that do not have price data:

    \begin{Verbatim}[commandchars=\\\{\}]
{\color{incolor}In [{\color{incolor} }]:} \PY{c+c1}{\PYZsh{} simply drop whole row with NaN in \PYZdq{}price\PYZdq{} column}
        \PY{n}{df}\PY{o}{.}\PY{n}{dropna}\PY{p}{(}\PY{n}{subset}\PY{o}{=}\PY{p}{[}\PY{l+s+s2}{\PYZdq{}}\PY{l+s+s2}{price}\PY{l+s+s2}{\PYZdq{}}\PY{p}{]}\PY{p}{,} \PY{n}{axis}\PY{o}{=}\PY{l+m+mi}{0}\PY{p}{,} \PY{n}{inplace}\PY{o}{=}\PY{k+kc}{True}\PY{p}{)}
        
        \PY{c+c1}{\PYZsh{} reset index, because we droped two rows}
        \PY{n}{df}\PY{o}{.}\PY{n}{reset\PYZus{}index}\PY{p}{(}\PY{n}{drop}\PY{o}{=}\PY{k+kc}{True}\PY{p}{,} \PY{n}{inplace}\PY{o}{=}\PY{k+kc}{True}\PY{p}{)}
\end{Verbatim}

    \begin{Verbatim}[commandchars=\\\{\}]
{\color{incolor}In [{\color{incolor} }]:} \PY{n}{df}\PY{o}{.}\PY{n}{head}\PY{p}{(}\PY{p}{)}
\end{Verbatim}

    Good! Now, we obtain the dataset with no missing values.

    Correct data format

We are almost there!

The last step in data cleaning is checking and making sure that all data
is in the correct format (int, float, text or other).

In Pandas, we use

.dtype() to check the data type

.astype() to change the data type

    Lets list the data types for each column

    \begin{Verbatim}[commandchars=\\\{\}]
{\color{incolor}In [{\color{incolor} }]:} \PY{n}{df}\PY{o}{.}\PY{n}{dtypes}
\end{Verbatim}

    As we can see above, some columns are not of the correct data type.
Numerical variables should have type 'float' or 'int', and variables
with strings such as categories should have type 'object'. For example,
'bore' and 'stroke' variables are numerical values that describe the
engines, so we should expect them to be of the type 'float' or 'int';
however, they are shown as type 'object'. We have to convert data types
into a proper format for each column using the "astype()" method.

    Convert data types to proper format

    \begin{Verbatim}[commandchars=\\\{\}]
{\color{incolor}In [{\color{incolor} }]:} \PY{n}{df}\PY{p}{[}\PY{p}{[}\PY{l+s+s2}{\PYZdq{}}\PY{l+s+s2}{bore}\PY{l+s+s2}{\PYZdq{}}\PY{p}{,} \PY{l+s+s2}{\PYZdq{}}\PY{l+s+s2}{stroke}\PY{l+s+s2}{\PYZdq{}}\PY{p}{]}\PY{p}{]} \PY{o}{=} \PY{n}{df}\PY{p}{[}\PY{p}{[}\PY{l+s+s2}{\PYZdq{}}\PY{l+s+s2}{bore}\PY{l+s+s2}{\PYZdq{}}\PY{p}{,} \PY{l+s+s2}{\PYZdq{}}\PY{l+s+s2}{stroke}\PY{l+s+s2}{\PYZdq{}}\PY{p}{]}\PY{p}{]}\PY{o}{.}\PY{n}{astype}\PY{p}{(}\PY{l+s+s2}{\PYZdq{}}\PY{l+s+s2}{float}\PY{l+s+s2}{\PYZdq{}}\PY{p}{)}
        \PY{n}{df}\PY{p}{[}\PY{p}{[}\PY{l+s+s2}{\PYZdq{}}\PY{l+s+s2}{normalized\PYZhy{}losses}\PY{l+s+s2}{\PYZdq{}}\PY{p}{]}\PY{p}{]} \PY{o}{=} \PY{n}{df}\PY{p}{[}\PY{p}{[}\PY{l+s+s2}{\PYZdq{}}\PY{l+s+s2}{normalized\PYZhy{}losses}\PY{l+s+s2}{\PYZdq{}}\PY{p}{]}\PY{p}{]}\PY{o}{.}\PY{n}{astype}\PY{p}{(}\PY{l+s+s2}{\PYZdq{}}\PY{l+s+s2}{int}\PY{l+s+s2}{\PYZdq{}}\PY{p}{)}
        \PY{n}{df}\PY{p}{[}\PY{p}{[}\PY{l+s+s2}{\PYZdq{}}\PY{l+s+s2}{price}\PY{l+s+s2}{\PYZdq{}}\PY{p}{]}\PY{p}{]} \PY{o}{=} \PY{n}{df}\PY{p}{[}\PY{p}{[}\PY{l+s+s2}{\PYZdq{}}\PY{l+s+s2}{price}\PY{l+s+s2}{\PYZdq{}}\PY{p}{]}\PY{p}{]}\PY{o}{.}\PY{n}{astype}\PY{p}{(}\PY{l+s+s2}{\PYZdq{}}\PY{l+s+s2}{float}\PY{l+s+s2}{\PYZdq{}}\PY{p}{)}
        \PY{n}{df}\PY{p}{[}\PY{p}{[}\PY{l+s+s2}{\PYZdq{}}\PY{l+s+s2}{peak\PYZhy{}rpm}\PY{l+s+s2}{\PYZdq{}}\PY{p}{]}\PY{p}{]} \PY{o}{=} \PY{n}{df}\PY{p}{[}\PY{p}{[}\PY{l+s+s2}{\PYZdq{}}\PY{l+s+s2}{peak\PYZhy{}rpm}\PY{l+s+s2}{\PYZdq{}}\PY{p}{]}\PY{p}{]}\PY{o}{.}\PY{n}{astype}\PY{p}{(}\PY{l+s+s2}{\PYZdq{}}\PY{l+s+s2}{float}\PY{l+s+s2}{\PYZdq{}}\PY{p}{)}
\end{Verbatim}

    Let us list the columns after the conversion

    \begin{Verbatim}[commandchars=\\\{\}]
{\color{incolor}In [{\color{incolor} }]:} \PY{n}{df}\PY{o}{.}\PY{n}{dtypes}
\end{Verbatim}

    Wonderful!

Now, we finally obtain the cleaned dataset with no missing values and
all data in its proper format.

    Data Standardization

Data is usually collected from different agencies with different
formats. (Data Standardization is also a term for a particular type of
data normalization, where we subtract the mean and divide by the
standard deviation)

What is Standardization?

Standardization is the process of transforming data into a common format
which allows the researcher to make the meaningful comparison.

Example

Transform mpg to L/100km:

In our dataset, the fuel consumption columns "city-mpg" and
"highway-mpg" are represented by mpg (miles per gallon) unit. Assume we
are developing an application in a country that accept the fuel
consumption with L/100km standard

We will need to apply data transformation to transform mpg into L/100km?

    The formula for unit conversion is

L/100km = 235 / mpg

We can do many mathematical operations directly in Pandas.

    \begin{Verbatim}[commandchars=\\\{\}]
{\color{incolor}In [{\color{incolor} }]:} \PY{n}{df}\PY{o}{.}\PY{n}{head}\PY{p}{(}\PY{p}{)}
\end{Verbatim}

    \begin{Verbatim}[commandchars=\\\{\}]
{\color{incolor}In [{\color{incolor} }]:} \PY{c+c1}{\PYZsh{} Convert mpg to L/100km by mathematical operation (235 divided by mpg)}
        \PY{n}{df}\PY{p}{[}\PY{l+s+s1}{\PYZsq{}}\PY{l+s+s1}{city\PYZhy{}L/100km}\PY{l+s+s1}{\PYZsq{}}\PY{p}{]} \PY{o}{=} \PY{l+m+mi}{235}\PY{o}{/}\PY{n}{df}\PY{p}{[}\PY{l+s+s2}{\PYZdq{}}\PY{l+s+s2}{city\PYZhy{}mpg}\PY{l+s+s2}{\PYZdq{}}\PY{p}{]}
        
        \PY{c+c1}{\PYZsh{} check your transformed data }
        \PY{n}{df}\PY{o}{.}\PY{n}{head}\PY{p}{(}\PY{p}{)}
\end{Verbatim}

    Question \#2:

According to the example above, transform mpg to L/100km in the column
of "highway-mpg", and change the name of column to "highway-L/100km".

    \begin{Verbatim}[commandchars=\\\{\}]
{\color{incolor}In [{\color{incolor} }]:} \PY{c+c1}{\PYZsh{} Write your code below and press Shift+Enter to execute }
\end{Verbatim}

    Double-click here for the solution.

    Data Normalization

Why normalization?

Normalization is the process of transforming values of several variables
into a similar range. Typical normalizations include scaling the
variable so the variable average is 0, scaling the variable so the
variance is 1, or scaling variable so the variable values range from 0
to 1

Example

To demonstrate normalization, let's say we want to scale the columns
"length", "width" and "height"

Target:would like to Normalize those variables so their value ranges
from 0 to 1.

Approach: replace original value by (original value)/(maximum value)

    \begin{Verbatim}[commandchars=\\\{\}]
{\color{incolor}In [{\color{incolor} }]:} \PY{c+c1}{\PYZsh{} replace (original value) by (original value)/(maximum value)}
        \PY{n}{df}\PY{p}{[}\PY{l+s+s1}{\PYZsq{}}\PY{l+s+s1}{length}\PY{l+s+s1}{\PYZsq{}}\PY{p}{]} \PY{o}{=} \PY{n}{df}\PY{p}{[}\PY{l+s+s1}{\PYZsq{}}\PY{l+s+s1}{length}\PY{l+s+s1}{\PYZsq{}}\PY{p}{]}\PY{o}{/}\PY{n}{df}\PY{p}{[}\PY{l+s+s1}{\PYZsq{}}\PY{l+s+s1}{length}\PY{l+s+s1}{\PYZsq{}}\PY{p}{]}\PY{o}{.}\PY{n}{max}\PY{p}{(}\PY{p}{)}
        \PY{n}{df}\PY{p}{[}\PY{l+s+s1}{\PYZsq{}}\PY{l+s+s1}{width}\PY{l+s+s1}{\PYZsq{}}\PY{p}{]} \PY{o}{=} \PY{n}{df}\PY{p}{[}\PY{l+s+s1}{\PYZsq{}}\PY{l+s+s1}{width}\PY{l+s+s1}{\PYZsq{}}\PY{p}{]}\PY{o}{/}\PY{n}{df}\PY{p}{[}\PY{l+s+s1}{\PYZsq{}}\PY{l+s+s1}{width}\PY{l+s+s1}{\PYZsq{}}\PY{p}{]}\PY{o}{.}\PY{n}{max}\PY{p}{(}\PY{p}{)}
\end{Verbatim}

    Questiont \#3:

According to the example above, normalize the column "height".

    \begin{Verbatim}[commandchars=\\\{\}]
{\color{incolor}In [{\color{incolor} }]:} \PY{c+c1}{\PYZsh{} Write your code below and press Shift+Enter to execute }
\end{Verbatim}

    Double-click here for the solution.

    Here we can see, we've normalized "length", "width" and "height" in the
range of {[}0,1{]}.

    Binning

Why binning?

\begin{verbatim}
Binning is a process of transforming continuous numerical variables into discrete categorical 'bins', for grouped analysis.
\end{verbatim}

Example:

In our dataset, "horsepower" is a real valued variable ranging from 48
to 288, it has 57 unique values. What if we only care about the price
difference between cars with high horsepower, medium horsepower, and
little horsepower (3 types)? Can we rearrange them into three `bins' to
simplify analysis?

We will use the Pandas method 'cut' to segment the 'horsepower' column
into 3 bins

    Example of Binning Data In Pandas

    Convert data to correct format

    \begin{Verbatim}[commandchars=\\\{\}]
{\color{incolor}In [{\color{incolor} }]:} \PY{n}{df}\PY{p}{[}\PY{l+s+s2}{\PYZdq{}}\PY{l+s+s2}{horsepower}\PY{l+s+s2}{\PYZdq{}}\PY{p}{]}\PY{o}{=}\PY{n}{df}\PY{p}{[}\PY{l+s+s2}{\PYZdq{}}\PY{l+s+s2}{horsepower}\PY{l+s+s2}{\PYZdq{}}\PY{p}{]}\PY{o}{.}\PY{n}{astype}\PY{p}{(}\PY{n+nb}{int}\PY{p}{,} \PY{n}{copy}\PY{o}{=}\PY{k+kc}{True}\PY{p}{)}
\end{Verbatim}

    Lets plot the histogram of horspower, to see what the distribution of
horsepower looks like.

    \begin{Verbatim}[commandchars=\\\{\}]
{\color{incolor}In [{\color{incolor} }]:} \PY{o}{\PYZpc{}}\PY{k}{matplotlib} inline
        \PY{k+kn}{import} \PY{n+nn}{matplotlib} \PY{k}{as} \PY{n+nn}{plt}
        \PY{k+kn}{from} \PY{n+nn}{matplotlib} \PY{k}{import} \PY{n}{pyplot}
        \PY{n}{plt}\PY{o}{.}\PY{n}{pyplot}\PY{o}{.}\PY{n}{hist}\PY{p}{(}\PY{n}{df}\PY{p}{[}\PY{l+s+s2}{\PYZdq{}}\PY{l+s+s2}{horsepower}\PY{l+s+s2}{\PYZdq{}}\PY{p}{]}\PY{p}{)}
        
        \PY{c+c1}{\PYZsh{} set x/y labels and plot title}
        \PY{n}{plt}\PY{o}{.}\PY{n}{pyplot}\PY{o}{.}\PY{n}{xlabel}\PY{p}{(}\PY{l+s+s2}{\PYZdq{}}\PY{l+s+s2}{horsepower}\PY{l+s+s2}{\PYZdq{}}\PY{p}{)}
        \PY{n}{plt}\PY{o}{.}\PY{n}{pyplot}\PY{o}{.}\PY{n}{ylabel}\PY{p}{(}\PY{l+s+s2}{\PYZdq{}}\PY{l+s+s2}{count}\PY{l+s+s2}{\PYZdq{}}\PY{p}{)}
        \PY{n}{plt}\PY{o}{.}\PY{n}{pyplot}\PY{o}{.}\PY{n}{title}\PY{p}{(}\PY{l+s+s2}{\PYZdq{}}\PY{l+s+s2}{horsepower bins}\PY{l+s+s2}{\PYZdq{}}\PY{p}{)}
\end{Verbatim}

    We would like 3 bins of equal size bandwidth so we use numpy's
linspace(start\_value, end\_value, numbers\_generated function.

Since we want to include the minimum value of horsepower we want to set
start\_value=min(df{[}"horsepower"{]}).

Since we want to include the maximum value of horsepower we want to set
end\_value=max(df{[}"horsepower"{]}).

Since we are building 3 bins of equal length, there should be 4
dividers, so numbers\_generated=4.

    We build a bin array, with a minimum value to a maximum value, with
bandwidth calculated above. The bins will be values used to determine
when one bin ends and another begins.

    \begin{Verbatim}[commandchars=\\\{\}]
{\color{incolor}In [{\color{incolor} }]:} \PY{n}{bins} \PY{o}{=} \PY{n}{np}\PY{o}{.}\PY{n}{linspace}\PY{p}{(}\PY{n+nb}{min}\PY{p}{(}\PY{n}{df}\PY{p}{[}\PY{l+s+s2}{\PYZdq{}}\PY{l+s+s2}{horsepower}\PY{l+s+s2}{\PYZdq{}}\PY{p}{]}\PY{p}{)}\PY{p}{,} \PY{n+nb}{max}\PY{p}{(}\PY{n}{df}\PY{p}{[}\PY{l+s+s2}{\PYZdq{}}\PY{l+s+s2}{horsepower}\PY{l+s+s2}{\PYZdq{}}\PY{p}{]}\PY{p}{)}\PY{p}{,} \PY{l+m+mi}{4}\PY{p}{)}
        \PY{n}{bins}
\end{Verbatim}

    We set group names:

    \begin{Verbatim}[commandchars=\\\{\}]
{\color{incolor}In [{\color{incolor} }]:} \PY{n}{group\PYZus{}names} \PY{o}{=} \PY{p}{[}\PY{l+s+s1}{\PYZsq{}}\PY{l+s+s1}{Low}\PY{l+s+s1}{\PYZsq{}}\PY{p}{,} \PY{l+s+s1}{\PYZsq{}}\PY{l+s+s1}{Medium}\PY{l+s+s1}{\PYZsq{}}\PY{p}{,} \PY{l+s+s1}{\PYZsq{}}\PY{l+s+s1}{High}\PY{l+s+s1}{\PYZsq{}}\PY{p}{]}
\end{Verbatim}

    We apply the function "cut" the determine what each value of
"df{[}'horsepower'{]}" belongs to.

    \begin{Verbatim}[commandchars=\\\{\}]
{\color{incolor}In [{\color{incolor} }]:} \PY{n}{df}\PY{p}{[}\PY{l+s+s1}{\PYZsq{}}\PY{l+s+s1}{horsepower\PYZhy{}binned}\PY{l+s+s1}{\PYZsq{}}\PY{p}{]} \PY{o}{=} \PY{n}{pd}\PY{o}{.}\PY{n}{cut}\PY{p}{(}\PY{n}{df}\PY{p}{[}\PY{l+s+s1}{\PYZsq{}}\PY{l+s+s1}{horsepower}\PY{l+s+s1}{\PYZsq{}}\PY{p}{]}\PY{p}{,} \PY{n}{bins}\PY{p}{,} \PY{n}{labels}\PY{o}{=}\PY{n}{group\PYZus{}names}\PY{p}{,} \PY{n}{include\PYZus{}lowest}\PY{o}{=}\PY{k+kc}{True} \PY{p}{)}
        \PY{n}{df}\PY{p}{[}\PY{p}{[}\PY{l+s+s1}{\PYZsq{}}\PY{l+s+s1}{horsepower}\PY{l+s+s1}{\PYZsq{}}\PY{p}{,}\PY{l+s+s1}{\PYZsq{}}\PY{l+s+s1}{horsepower\PYZhy{}binned}\PY{l+s+s1}{\PYZsq{}}\PY{p}{]}\PY{p}{]}\PY{o}{.}\PY{n}{head}\PY{p}{(}\PY{l+m+mi}{20}\PY{p}{)}
\end{Verbatim}

    Lets see the number of vehicles in each bin.

    \begin{Verbatim}[commandchars=\\\{\}]
{\color{incolor}In [{\color{incolor} }]:} \PY{n}{df}\PY{p}{[}\PY{l+s+s2}{\PYZdq{}}\PY{l+s+s2}{horsepower\PYZhy{}binned}\PY{l+s+s2}{\PYZdq{}}\PY{p}{]}\PY{o}{.}\PY{n}{value\PYZus{}counts}\PY{p}{(}\PY{p}{)}
\end{Verbatim}

    Lets plot the distribution of each bin.

    \begin{Verbatim}[commandchars=\\\{\}]
{\color{incolor}In [{\color{incolor} }]:} \PY{o}{\PYZpc{}}\PY{k}{matplotlib} inline
        \PY{k+kn}{import} \PY{n+nn}{matplotlib} \PY{k}{as} \PY{n+nn}{plt}
        \PY{k+kn}{from} \PY{n+nn}{matplotlib} \PY{k}{import} \PY{n}{pyplot}
        \PY{n}{pyplot}\PY{o}{.}\PY{n}{bar}\PY{p}{(}\PY{n}{group\PYZus{}names}\PY{p}{,} \PY{n}{df}\PY{p}{[}\PY{l+s+s2}{\PYZdq{}}\PY{l+s+s2}{horsepower\PYZhy{}binned}\PY{l+s+s2}{\PYZdq{}}\PY{p}{]}\PY{o}{.}\PY{n}{value\PYZus{}counts}\PY{p}{(}\PY{p}{)}\PY{p}{)}
        
        \PY{c+c1}{\PYZsh{} set x/y labels and plot title}
        \PY{n}{plt}\PY{o}{.}\PY{n}{pyplot}\PY{o}{.}\PY{n}{xlabel}\PY{p}{(}\PY{l+s+s2}{\PYZdq{}}\PY{l+s+s2}{horsepower}\PY{l+s+s2}{\PYZdq{}}\PY{p}{)}
        \PY{n}{plt}\PY{o}{.}\PY{n}{pyplot}\PY{o}{.}\PY{n}{ylabel}\PY{p}{(}\PY{l+s+s2}{\PYZdq{}}\PY{l+s+s2}{count}\PY{l+s+s2}{\PYZdq{}}\PY{p}{)}
        \PY{n}{plt}\PY{o}{.}\PY{n}{pyplot}\PY{o}{.}\PY{n}{title}\PY{p}{(}\PY{l+s+s2}{\PYZdq{}}\PY{l+s+s2}{horsepower bins}\PY{l+s+s2}{\PYZdq{}}\PY{p}{)}
\end{Verbatim}

    \begin{verbatim}
Check the dataframe above carefully, you will find the last column provides the bins for "horsepower" with 3 categories ("Low","Medium" and "High"). 
\end{verbatim}

\begin{verbatim}
We successfully narrow the intervals from 57 to 3!
\end{verbatim}

    Bins visualization

Normally, a histogram is used to visualize the distribution of bins we
created above.

    \begin{Verbatim}[commandchars=\\\{\}]
{\color{incolor}In [{\color{incolor} }]:} \PY{o}{\PYZpc{}}\PY{k}{matplotlib} inline
        \PY{k+kn}{import} \PY{n+nn}{matplotlib} \PY{k}{as} \PY{n+nn}{plt}
        \PY{k+kn}{from} \PY{n+nn}{matplotlib} \PY{k}{import} \PY{n}{pyplot}
        
        \PY{n}{a} \PY{o}{=} \PY{p}{(}\PY{l+m+mi}{0}\PY{p}{,}\PY{l+m+mi}{1}\PY{p}{,}\PY{l+m+mi}{2}\PY{p}{)}
        
        \PY{c+c1}{\PYZsh{} draw historgram of attribute \PYZdq{}horsepower\PYZdq{} with bins = 3}
        \PY{n}{plt}\PY{o}{.}\PY{n}{pyplot}\PY{o}{.}\PY{n}{hist}\PY{p}{(}\PY{n}{df}\PY{p}{[}\PY{l+s+s2}{\PYZdq{}}\PY{l+s+s2}{horsepower}\PY{l+s+s2}{\PYZdq{}}\PY{p}{]}\PY{p}{,} \PY{n}{bins} \PY{o}{=} \PY{l+m+mi}{3}\PY{p}{)}
        
        \PY{c+c1}{\PYZsh{} set x/y labels and plot title}
        \PY{n}{plt}\PY{o}{.}\PY{n}{pyplot}\PY{o}{.}\PY{n}{xlabel}\PY{p}{(}\PY{l+s+s2}{\PYZdq{}}\PY{l+s+s2}{horsepower}\PY{l+s+s2}{\PYZdq{}}\PY{p}{)}
        \PY{n}{plt}\PY{o}{.}\PY{n}{pyplot}\PY{o}{.}\PY{n}{ylabel}\PY{p}{(}\PY{l+s+s2}{\PYZdq{}}\PY{l+s+s2}{count}\PY{l+s+s2}{\PYZdq{}}\PY{p}{)}
        \PY{n}{plt}\PY{o}{.}\PY{n}{pyplot}\PY{o}{.}\PY{n}{title}\PY{p}{(}\PY{l+s+s2}{\PYZdq{}}\PY{l+s+s2}{horsepower bins}\PY{l+s+s2}{\PYZdq{}}\PY{p}{)}
\end{Verbatim}

    The plot above shows the binning result for attribute "horsepower".

    Indicator variable (or dummy variable)

What is an indicator variable?

\begin{verbatim}
An indicator variable (or dummy variable) is a numerical variable used to label categories. They are called 'dummies' because the numbers themselves don't have inherent meaning. 
\end{verbatim}

Why we use indicator variables?

\begin{verbatim}
So we can use categorical variables for regression analysis in the later modules.
\end{verbatim}

Example

\begin{verbatim}
We see the column "fuel-type" has two unique values, "gas" or "diesel". Regression doesn't understand words, only numbers. To use this attribute in regression analysis, we convert "fuel-type" into indicator variables.
\end{verbatim}

\begin{verbatim}
We will use the panda's method 'get_dummies' to assign numerical values to different categories of fuel type. 
\end{verbatim}

    \begin{Verbatim}[commandchars=\\\{\}]
{\color{incolor}In [{\color{incolor} }]:} \PY{n}{df}\PY{o}{.}\PY{n}{columns}
\end{Verbatim}

    get indicator variables and assign it to data frame "dummy\_variable\_1"

    \begin{Verbatim}[commandchars=\\\{\}]
{\color{incolor}In [{\color{incolor} }]:} \PY{n}{dummy\PYZus{}variable\PYZus{}1} \PY{o}{=} \PY{n}{pd}\PY{o}{.}\PY{n}{get\PYZus{}dummies}\PY{p}{(}\PY{n}{df}\PY{p}{[}\PY{l+s+s2}{\PYZdq{}}\PY{l+s+s2}{fuel\PYZhy{}type}\PY{l+s+s2}{\PYZdq{}}\PY{p}{]}\PY{p}{)}
        \PY{n}{dummy\PYZus{}variable\PYZus{}1}\PY{o}{.}\PY{n}{head}\PY{p}{(}\PY{p}{)}
\end{Verbatim}

    change column names for clarity

    \begin{Verbatim}[commandchars=\\\{\}]
{\color{incolor}In [{\color{incolor} }]:} \PY{n}{dummy\PYZus{}variable\PYZus{}1}\PY{o}{.}\PY{n}{rename}\PY{p}{(}\PY{n}{columns}\PY{o}{=}\PY{p}{\PYZob{}}\PY{l+s+s1}{\PYZsq{}}\PY{l+s+s1}{fuel\PYZhy{}type\PYZhy{}diesel}\PY{l+s+s1}{\PYZsq{}}\PY{p}{:}\PY{l+s+s1}{\PYZsq{}}\PY{l+s+s1}{gas}\PY{l+s+s1}{\PYZsq{}}\PY{p}{,} \PY{l+s+s1}{\PYZsq{}}\PY{l+s+s1}{fuel\PYZhy{}type\PYZhy{}diesel}\PY{l+s+s1}{\PYZsq{}}\PY{p}{:}\PY{l+s+s1}{\PYZsq{}}\PY{l+s+s1}{diesel}\PY{l+s+s1}{\PYZsq{}}\PY{p}{\PYZcb{}}\PY{p}{,} \PY{n}{inplace}\PY{o}{=}\PY{k+kc}{True}\PY{p}{)}
        \PY{n}{dummy\PYZus{}variable\PYZus{}1}\PY{o}{.}\PY{n}{head}\PY{p}{(}\PY{p}{)}
\end{Verbatim}

    We now have the value 0 to represent "gas" and 1 to represent "diesel"
in the column "fuel-type". We will now insert this column back into our
original dataset.

    \begin{Verbatim}[commandchars=\\\{\}]
{\color{incolor}In [{\color{incolor} }]:} \PY{c+c1}{\PYZsh{} merge data frame \PYZdq{}df\PYZdq{} and \PYZdq{}dummy\PYZus{}variable\PYZus{}1\PYZdq{} }
        \PY{n}{df} \PY{o}{=} \PY{n}{pd}\PY{o}{.}\PY{n}{concat}\PY{p}{(}\PY{p}{[}\PY{n}{df}\PY{p}{,} \PY{n}{dummy\PYZus{}variable\PYZus{}1}\PY{p}{]}\PY{p}{,} \PY{n}{axis}\PY{o}{=}\PY{l+m+mi}{1}\PY{p}{)}
        
        \PY{c+c1}{\PYZsh{} drop original column \PYZdq{}fuel\PYZhy{}type\PYZdq{} from \PYZdq{}df\PYZdq{}}
        \PY{n}{df}\PY{o}{.}\PY{n}{drop}\PY{p}{(}\PY{l+s+s2}{\PYZdq{}}\PY{l+s+s2}{fuel\PYZhy{}type}\PY{l+s+s2}{\PYZdq{}}\PY{p}{,} \PY{n}{axis} \PY{o}{=} \PY{l+m+mi}{1}\PY{p}{,} \PY{n}{inplace}\PY{o}{=}\PY{k+kc}{True}\PY{p}{)}
\end{Verbatim}

    \begin{Verbatim}[commandchars=\\\{\}]
{\color{incolor}In [{\color{incolor} }]:} \PY{n}{df}\PY{o}{.}\PY{n}{head}\PY{p}{(}\PY{p}{)}
\end{Verbatim}

    The last two columns are now the indicator variable representation of
the fuel-type variable. It's all 0s and 1s now.

    Question \#4:

As above, create indicator variable to the column of "aspiration": "std"
to 0, while "turbo" to 1.

    \begin{Verbatim}[commandchars=\\\{\}]
{\color{incolor}In [{\color{incolor} }]:} \PY{c+c1}{\PYZsh{} Write your code below and press Shift+Enter to execute }
\end{Verbatim}

    Double-click here for the solution.

    Question \#5:

Merge the new dataframe to the original dataframe then drop the column
'aspiration'

    \begin{Verbatim}[commandchars=\\\{\}]
{\color{incolor}In [{\color{incolor} }]:} \PY{c+c1}{\PYZsh{} Write your code below and press Shift+Enter to execute }
\end{Verbatim}

    Double-click here for the solution.

    save the new csv

    \begin{Verbatim}[commandchars=\\\{\}]
{\color{incolor}In [{\color{incolor} }]:} \PY{n}{df}\PY{o}{.}\PY{n}{to\PYZus{}csv}\PY{p}{(}\PY{l+s+s1}{\PYZsq{}}\PY{l+s+s1}{clean\PYZus{}df.csv}\PY{l+s+s1}{\PYZsq{}}\PY{p}{)}
\end{Verbatim}

    Thank you for completing this notebook

    \begin{verbatim}
<p><a href="https://cocl.us/corsera_da0101en_notebook_bottom"><img src="https://s3-api.us-geo.objectstorage.softlayer.net/cf-courses-data/CognitiveClass/DA0101EN/Images/BottomAd.png" width="750" align="center"></a></p>
\end{verbatim}

    About the Authors:

This notebook was written by Mahdi Noorian PhD, Joseph Santarcangelo,
Bahare Talayian, Eric Xiao, Steven Dong, Parizad, Hima Vsudevan and
Fiorella Wenver and Yi Yao.

Joseph Santarcangelo is a Data Scientist at IBM, and holds a PhD in
Electrical Engineering. His research focused on using Machine Learning,
Signal Processing, and Computer Vision to determine how videos impact
human cognition. Joseph has been working for IBM since he completed his
PhD.

    Copyright © 2018 IBM Developer Skills Network. This notebook and its
source code are released under the terms of the MIT License.


    % Add a bibliography block to the postdoc
    
    
    
    \end{document}
